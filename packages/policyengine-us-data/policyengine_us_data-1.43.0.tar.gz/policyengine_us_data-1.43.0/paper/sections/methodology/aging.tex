\subsection{Data Aging and Indexing}

The process of projecting historical microdata involves both demographic aging and economic indexing based on US government forecasts. Our aging process occurs in two stages: first to reach our baseline year (2024), and then to project the calibrated dataset forward.

\subsubsection{Growth Factor Construction}

For each variable in the tax-benefit system with a specified growth parameter, we compute change factors from the base year through 2034:

\[ \text{Index Factor}_{t} = \frac{\text{Index}_{t}}{\text{Index}_{\text{base}}} \]

\subsubsection{Population Adjustment}

Most economic variables are adjusted for changes in total population:

\[ \text{Per Capita Factor}_{t} = \frac{\text{Index Factor}_{t}}{\text{Population Growth}_{t}} \]

Exceptions include:
\begin{itemize}
    \item Weight variables maintain raw growth
    \item Population itself uses Census projections directly
\end{itemize}

\subsubsection{Data Sources}

Projection factors come from:
\begin{itemize}
    \item Congressional Budget Office economic projections
    \item Census Bureau population estimates 
    \item Social Security Administration wage index forecasts
    \item Treasury tax parameter indexing
\end{itemize}

\subsubsection{Initial Aging Implementation}

For any variable y, the projected value to reach our baseline year is computed as:

\[ y_{2024} = y_{2023} \cdot \frac{f(2024)}{f(2023)} \]

where f(t) represents the index factor for time t.

\subsubsection{Forward Projection}

After constructing and calibrating the enhanced 2024 dataset, we project it to future years using the same indexing framework. This maintains the dataset's enhanced distributional properties while reflecting:

\begin{itemize}
    \item Economic growth forecasts for monetary variables
    \item Statutory adjustments to program parameters
    \item Population projections applied to household weights
\end{itemize}