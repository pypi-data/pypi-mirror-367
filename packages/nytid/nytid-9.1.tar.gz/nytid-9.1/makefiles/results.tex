\section{Introduction and usage}% ===> this file was generated automatically by noweave --- better not edit it
\label{Intro}

The problem case is the following.
We have a Moodle system where we do grading and everything related to a course, 
\ie we have individual assignments.
Then we must report the grades to a national database.
The entries in this database is according to parts set in the course syllabus, 
each part can contain one or more assignments.
This makefile uses the data that can be extracted from Moodle and some 
settings, then it converts the data to a form which is reportable to the 
student office, where the report is \emph{manually} entered into the national 
database.

Since the data must be manually entered into the database, we require that the 
reports we send are not overlapping.
E.g.\ if we report all grades by the end of a course, but some students 
complete their assignments late and are graded after the first report, then we 
must generate a second report which only contains the new results.

The input is a file which is exported from Moodle (a tab-separated CSV-file).
The output is also a tab-separated CSV-file, reflecting the current state of 
what has been reported to the national database.
The output could thus simply be a copy of the input, it will be used for 
comparison the next time we generate a report.
Finally, we will output a temporary file, the report to be sent to the student 
office for registration.
For this we will use three variables that can be set on the command-line:
\nwfilename{results.mk.nw}\nwbegincode{1}\sublabel{NW1vYGPv-29jjeg-1}\nwmargintag{{\nwtagstyle{}\subpageref{NW1vYGPv-29jjeg-1}}}\moddef{variables~{\nwtagstyle{}\subpageref{NW1vYGPv-29jjeg-1}}}\endmoddef\nwstartdeflinemarkup\nwusesondefline{\\{NW1vYGPv-4UwcQh-1}}\nwprevnextdefs{\relax}{NW1vYGPv-29jjeg-2}\nwenddeflinemarkup
in?=      new.csv
out?=     reported.csv
report?=  report.pdf
\nwalsodefined{\\{NW1vYGPv-29jjeg-2}\\{NW1vYGPv-29jjeg-3}\\{NW1vYGPv-29jjeg-4}\\{NW1vYGPv-29jjeg-5}\\{NW1vYGPv-29jjeg-6}\\{NW1vYGPv-29jjeg-7}}\nwused{\\{NW1vYGPv-4UwcQh-1}}\nwendcode{}\nwbegindocs{2}\nwdocspar

The structure will be that of a makefile used for inclusion in a main 
Makefile.
The structure is thus similar to most makefiles, we first need 
{\Tt{}\LA{}variables~{\nwtagstyle{}\subpageref{NW1vYGPv-29jjeg-1}}\RA{}\nwendquote}, then {\Tt{}\LA{}targets~{\nwtagstyle{}\subpageref{NW1vYGPv-2InVV3-1}}\RA{}\nwendquote}.
Since this will be a file to include, we do not want to include the same 
contents twice, in any form of accidental recursive inclusion, so we use 
a C-like construction.
\nwenddocs{}\nwbegincode{3}\sublabel{NW1vYGPv-4UwcQh-1}\nwmargintag{{\nwtagstyle{}\subpageref{NW1vYGPv-4UwcQh-1}}}\moddef{results.mk~{\nwtagstyle{}\subpageref{NW1vYGPv-4UwcQh-1}}}\endmoddef\nwstartdeflinemarkup\nwenddeflinemarkup
ifndef MIUN_RESULTS_MK
MIUN_RESULTS_MK=true

\LA{}variables~{\nwtagstyle{}\subpageref{NW1vYGPv-29jjeg-1}}\RA{}
\LA{}targets~{\nwtagstyle{}\subpageref{NW1vYGPv-2InVV3-1}}\RA{}

INCLUDE_MAKEFILES?= .
include $\{INCLUDE_MAKEFILES\}/miun.depend.mk

endif
\nwnotused{results.mk}\nwendcode{}\nwbegindocs{4}So to use this file, simply input it in your Makefile by adding the line 
{\Tt{}include\ results.mk\nwendquote} at the end of the file, in the same fashion as the 
inclusion of {\Tt{}miun.depend.mk\nwendquote} above.

For the purpose of reporting the results, we need to provide some identifiers.
Usually this comes in the form of a course identifier.
We also need to know where to send the results, so we can automate as much as 
possible.
We will dedicate two variables for this, which can be set in a Makefile.
\nwenddocs{}\nwbegincode{5}\sublabel{NW1vYGPv-29jjeg-2}\nwmargintag{{\nwtagstyle{}\subpageref{NW1vYGPv-29jjeg-2}}}\moddef{variables~{\nwtagstyle{}\subpageref{NW1vYGPv-29jjeg-1}}}\plusendmoddef\nwstartdeflinemarkup\nwusesondefline{\\{NW1vYGPv-4UwcQh-1}}\nwprevnextdefs{NW1vYGPv-29jjeg-1}{NW1vYGPv-29jjeg-3}\nwenddeflinemarkup
RESULTS_COURSE?=    course identifier
RESULTS_EMAIL?=     iksexp@miun.se
\nwused{\\{NW1vYGPv-4UwcQh-1}}\nwendcode{}\nwbegindocs{6}These variables are later used to form the command for sending the results to
the student office.
By default we use Mutt\footnote{%
  We could also use Thunderbird by setting {\Tt{}RESULTS{\_}MAILER?=\ thunderbird\ -compose\ "to={\$}{\nwlbrace}RESULTS{\_}EMAIL{\nwrbrace},subject='resultat\ {\$}{\nwlbrace}RESULTS{\_}COURSE{\nwrbrace}',attachment='file://{\$}{\nwlbrace}report{\nwrbrace}'"\nwendquote}.
}.
\nwenddocs{}\nwbegincode{7}\sublabel{NW1vYGPv-29jjeg-3}\nwmargintag{{\nwtagstyle{}\subpageref{NW1vYGPv-29jjeg-3}}}\moddef{variables~{\nwtagstyle{}\subpageref{NW1vYGPv-29jjeg-1}}}\plusendmoddef\nwstartdeflinemarkup\nwusesondefline{\\{NW1vYGPv-4UwcQh-1}}\nwprevnextdefs{NW1vYGPv-29jjeg-2}{NW1vYGPv-29jjeg-4}\nwenddeflinemarkup
RESULTS_MAILER?=  mutt -s "resultat $\{RESULTS_COURSE\}" -a $\{report\} -- $\{RESULTS_EMAIL\}
\nwused{\\{NW1vYGPv-4UwcQh-1}}\nwendcode{}\nwbegindocs{8}\nwdocspar

We provide a target {\Tt{}report\nwendquote} which processes the input, generates the new 
report and emails it to the designated address above.
\nwenddocs{}\nwbegincode{9}\sublabel{NW1vYGPv-2InVV3-1}\nwmargintag{{\nwtagstyle{}\subpageref{NW1vYGPv-2InVV3-1}}}\moddef{targets~{\nwtagstyle{}\subpageref{NW1vYGPv-2InVV3-1}}}\endmoddef\nwstartdeflinemarkup\nwusesondefline{\\{NW1vYGPv-4UwcQh-1}}\nwprevnextdefs{\relax}{NW1vYGPv-2InVV3-2}\nwenddeflinemarkup
.PHONY: report
report:
  \LA{}report recipe~{\nwtagstyle{}\subpageref{NW1vYGPv-1bP5sY-1}}\RA{}
\nwalsodefined{\\{NW1vYGPv-2InVV3-2}\\{NW1vYGPv-2InVV3-3}\\{NW1vYGPv-2InVV3-4}\\{NW1vYGPv-2InVV3-5}\\{NW1vYGPv-2InVV3-6}\\{NW1vYGPv-2InVV3-7}\\{NW1vYGPv-2InVV3-8}\\{NW1vYGPv-2InVV3-9}\\{NW1vYGPv-2InVV3-A}\\{NW1vYGPv-2InVV3-B}}\nwused{\\{NW1vYGPv-4UwcQh-1}}\nwendcode{}\nwbegindocs{10}\nwdocspar

Finally, we also provide a way to clean up all temporary files.
We provide a target {\Tt{}clean-results\nwendquote} which we add as a dependency to the 
{\Tt{}clean\nwendquote} target, which is left to the user to use for whatever other cleaning 
is specified in the {\Tt{}Makefile\nwendquote}.
\nwenddocs{}\nwbegincode{11}\sublabel{NW1vYGPv-2InVV3-2}\nwmargintag{{\nwtagstyle{}\subpageref{NW1vYGPv-2InVV3-2}}}\moddef{targets~{\nwtagstyle{}\subpageref{NW1vYGPv-2InVV3-1}}}\plusendmoddef\nwstartdeflinemarkup\nwusesondefline{\\{NW1vYGPv-4UwcQh-1}}\nwprevnextdefs{NW1vYGPv-2InVV3-1}{NW1vYGPv-2InVV3-3}\nwenddeflinemarkup
.PHONY: clean clean-results
clean: clean-results
clean-results:
  \LA{}clean recipe~{\nwtagstyle{}\subpageref{NW1vYGPv-mtljB-1}}\RA{}
\nwused{\\{NW1vYGPv-4UwcQh-1}}\nwendcode{}\nwbegindocs{12}We will populate the {\Tt{}\LA{}clean recipe~{\nwtagstyle{}\subpageref{NW1vYGPv-mtljB-1}}\RA{}\nwendquote} as we go.

\subsection{Portability}

To improve the portability of the code, we use the following variables instead 
of the respective commands directly:
\nwenddocs{}\nwbegincode{13}\sublabel{NW1vYGPv-29jjeg-4}\nwmargintag{{\nwtagstyle{}\subpageref{NW1vYGPv-29jjeg-4}}}\moddef{variables~{\nwtagstyle{}\subpageref{NW1vYGPv-29jjeg-1}}}\plusendmoddef\nwstartdeflinemarkup\nwusesondefline{\\{NW1vYGPv-4UwcQh-1}}\nwprevnextdefs{NW1vYGPv-29jjeg-3}{NW1vYGPv-29jjeg-5}\nwenddeflinemarkup
LOCALC?=  localc --norestore
RM?=      /bin/rm -Rf
MV?=      /bin/mv
DIFF?=    diff
JOIN?=    join
CUT?=     cut
SORT?=    sort
HEAD?=    head
TAIL?=    tail
SED?=     sed
GREP?=    grep
CAT?=     cat
CP?=      cp -R
PAGER?=   less
PASTE?=   paste
LN?=      ln
\nwused{\\{NW1vYGPv-4UwcQh-1}}\nwendcode{}\nwbegindocs{14}There is currently an unknown bug causing the join command to not work with 
tabs, although that exact code has worked previously, so the resulting report 
file will be \emph{space separated}.


\section{Processing Moodle's output}

This section covers the technical details of how to process the data exported 
from Moodle.
We have the input file, given as {\Tt{}{\$}{\nwlbrace}in{\nwrbrace}\nwendquote}, then we want the a report of 
changes to send to the student office (\cref{FindingChanges}).
There are different identifiers used in the national database and in Moodle.
So we need to extract the identifiers in Moodle and convert to those in the 
national database (\cref{ChangingIdentifiers}).
Then we can send the report and update our local representation of what is 
reported to the national database, \ie {\Tt{}{\$}{\nwlbrace}out{\nwrbrace}\nwendquote}.

\subsection{Transforming Moodle's output}

The first thing we need to do is to transform Moodle's output.
The output format varies a lot, it changes with the mood of the system 
administrator.
So the code in this section changes the most.

We will now create a temporary file {\Tt{}{\$}{\nwlbrace}out{\nwrbrace}.diff\nwendquote} based on {\Tt{}{\$}{\nwlbrace}in{\nwrbrace}\nwendquote}.
\nwenddocs{}\nwbegincode{15}\sublabel{NW1vYGPv-2InVV3-3}\nwmargintag{{\nwtagstyle{}\subpageref{NW1vYGPv-2InVV3-3}}}\moddef{targets~{\nwtagstyle{}\subpageref{NW1vYGPv-2InVV3-1}}}\plusendmoddef\nwstartdeflinemarkup\nwusesondefline{\\{NW1vYGPv-4UwcQh-1}}\nwprevnextdefs{NW1vYGPv-2InVV3-2}{NW1vYGPv-2InVV3-4}\nwenddeflinemarkup
$\{out\}.new: $\{in\}
  \LA{}new recipe~{\nwtagstyle{}\subpageref{NW1vYGPv-HUskc-1}}\RA{}
\nwused{\\{NW1vYGPv-4UwcQh-1}}\nwendcode{}\nwbegindocs{16}This means we should also add {\Tt{}{\$}{\nwlbrace}out{\nwrbrace}.diff\nwendquote} to the recipe of clean.
\nwenddocs{}\nwbegincode{17}\sublabel{NW1vYGPv-mtljB-1}\nwmargintag{{\nwtagstyle{}\subpageref{NW1vYGPv-mtljB-1}}}\moddef{clean recipe~{\nwtagstyle{}\subpageref{NW1vYGPv-mtljB-1}}}\endmoddef\nwstartdeflinemarkup\nwusesondefline{\\{NW1vYGPv-2InVV3-2}}\nwprevnextdefs{\relax}{NW1vYGPv-mtljB-2}\nwenddeflinemarkup
$\{RM\} $\{out\}.new
\nwalsodefined{\\{NW1vYGPv-mtljB-2}\\{NW1vYGPv-mtljB-3}}\nwused{\\{NW1vYGPv-2InVV3-2}}\nwendcode{}\nwbegindocs{18}\nwdocspar

We are now going to process the data, we will do this by piping the data 
through a series of commands.
The columns we are interested in are 1--3 and 6 to the end.
\nwenddocs{}\nwbegincode{19}\sublabel{NW1vYGPv-HUskc-1}\nwmargintag{{\nwtagstyle{}\subpageref{NW1vYGPv-HUskc-1}}}\moddef{new recipe~{\nwtagstyle{}\subpageref{NW1vYGPv-HUskc-1}}}\endmoddef\nwstartdeflinemarkup\nwusesondefline{\\{NW1vYGPv-2InVV3-3}}\nwprevnextdefs{\relax}{NW1vYGPv-HUskc-2}\nwenddeflinemarkup
$\{CUT\} -f 1-3,6- $\{in\} | \\
\nwalsodefined{\\{NW1vYGPv-HUskc-2}\\{NW1vYGPv-HUskc-3}}\nwused{\\{NW1vYGPv-2InVV3-3}}\nwendcode{}\nwbegindocs{20}For some reason the students' usernames are appended to their lastnames --- 
in addition to having a separate column for usernames.
Obviously we want to filter this away.
\nwenddocs{}\nwbegincode{21}\sublabel{NW1vYGPv-HUskc-2}\nwmargintag{{\nwtagstyle{}\subpageref{NW1vYGPv-HUskc-2}}}\moddef{new recipe~{\nwtagstyle{}\subpageref{NW1vYGPv-HUskc-1}}}\plusendmoddef\nwstartdeflinemarkup\nwusesondefline{\\{NW1vYGPv-2InVV3-3}}\nwprevnextdefs{NW1vYGPv-HUskc-1}{NW1vYGPv-HUskc-3}\nwenddeflinemarkup
$\{SED\} "s/ (\\([a-z]\\\{4\\\}[0-9]\\\{4\\\}\\))//" \\
\nwused{\\{NW1vYGPv-2InVV3-3}}\nwendcode{}\nwbegindocs{22}\nwdocspar

Some of the data in Moodle are quite long, so we would like to do some 
rewrites.
For this purpose we will add a list of regular expressions that will be 
applied.
We store this list as a space separated list of regular expressions in 
{\Tt{}{\$}{\nwlbrace}RESULTS{\_}REWRITES{\nwrbrace}\nwendquote}.
This means that we also must avoid spaces in the regular expressions, thus the 
first thing we do is to remove all spaces in the data.
\nwenddocs{}\nwbegincode{23}\sublabel{NW1vYGPv-HUskc-3}\nwmargintag{{\nwtagstyle{}\subpageref{NW1vYGPv-HUskc-3}}}\moddef{new recipe~{\nwtagstyle{}\subpageref{NW1vYGPv-HUskc-1}}}\plusendmoddef\nwstartdeflinemarkup\nwusesondefline{\\{NW1vYGPv-2InVV3-3}}\nwprevnextdefs{NW1vYGPv-HUskc-2}{\relax}\nwenddeflinemarkup
$(if $\{RESULTS_REWRITES\},| $\{SED\} "s/ //g", ) \\
$(foreach regex,$\{RESULTS_REWRITES\},| $\{SED\} $\{regex\}) \\
> $@
\nwused{\\{NW1vYGPv-2InVV3-3}}\nwendcode{}\nwbegindocs{24}We let the default rewrites be
\nwenddocs{}\nwbegincode{25}\sublabel{NW1vYGPv-29jjeg-5}\nwmargintag{{\nwtagstyle{}\subpageref{NW1vYGPv-29jjeg-5}}}\moddef{variables~{\nwtagstyle{}\subpageref{NW1vYGPv-29jjeg-1}}}\plusendmoddef\nwstartdeflinemarkup\nwusesondefline{\\{NW1vYGPv-4UwcQh-1}}\nwprevnextdefs{NW1vYGPv-29jjeg-4}{NW1vYGPv-29jjeg-6}\nwenddeflinemarkup
RESULTS_REWRITES+=  "s/Godkänd(G)/G/g" "s/Underkänd(U)/U/g"
RESULTS_REWRITES+=  "s/Komplettering(Fx)/Fx/g"
RESULTS_REWRITES+=  "s/\\"//g"
\nwused{\\{NW1vYGPv-4UwcQh-1}}\nwendcode{}\nwbegindocs{26}\nwdocspar

\subsection{Extracting the changes}
\label{FindingChanges}

Now we want to find what has changed since the last time we exported the 
grades.
For this we will create a file {\Tt{}{\$}{\nwlbrace}out{\nwrbrace}.diff\nwendquote} which contains only the changed 
rows.
\nwenddocs{}\nwbegincode{27}\sublabel{NW1vYGPv-2InVV3-4}\nwmargintag{{\nwtagstyle{}\subpageref{NW1vYGPv-2InVV3-4}}}\moddef{targets~{\nwtagstyle{}\subpageref{NW1vYGPv-2InVV3-1}}}\plusendmoddef\nwstartdeflinemarkup\nwusesondefline{\\{NW1vYGPv-4UwcQh-1}}\nwprevnextdefs{NW1vYGPv-2InVV3-3}{NW1vYGPv-2InVV3-5}\nwenddeflinemarkup
$\{out\}.diff: $\{out\}.new
  \LA{}diff recipe~{\nwtagstyle{}\subpageref{NW1vYGPv-257BvH-1}}\RA{}
\nwused{\\{NW1vYGPv-4UwcQh-1}}\nwendcode{}\nwbegindocs{28}\nwdocspar

We are not interested in the headers of the table, so we skip that
line.
\nwenddocs{}\nwbegincode{29}\sublabel{NW1vYGPv-257BvH-1}\nwmargintag{{\nwtagstyle{}\subpageref{NW1vYGPv-257BvH-1}}}\moddef{diff recipe~{\nwtagstyle{}\subpageref{NW1vYGPv-257BvH-1}}}\endmoddef\nwstartdeflinemarkup\nwusesondefline{\\{NW1vYGPv-2InVV3-4}}\nwprevnextdefs{\relax}{NW1vYGPv-257BvH-2}\nwenddeflinemarkup
$\{GREP\} -v "^.\\?First \\?name" $\{out\}.new | \\
\nwalsodefined{\\{NW1vYGPv-257BvH-2}\\{NW1vYGPv-257BvH-3}}\nwused{\\{NW1vYGPv-2InVV3-4}}\nwendcode{}\nwbegindocs{30}We are not interested in reporting students' failed results, so we filter out 
those.
In {\Tt{}{\$}{\nwlbrace}RESULTS{\_}FAILED{\nwrbrace}\nwendquote} we keep a regular expression for the grades that shall
be removed.
We then use {\Tt{}{\$}{\nwlbrace}RESULTS{\_}FAILED{\_}regex{\nwrbrace}\nwendquote} to match against the data.
\nwenddocs{}\nwbegincode{31}\sublabel{NW1vYGPv-29jjeg-6}\nwmargintag{{\nwtagstyle{}\subpageref{NW1vYGPv-29jjeg-6}}}\moddef{variables~{\nwtagstyle{}\subpageref{NW1vYGPv-29jjeg-1}}}\plusendmoddef\nwstartdeflinemarkup\nwusesondefline{\\{NW1vYGPv-4UwcQh-1}}\nwprevnextdefs{NW1vYGPv-29jjeg-5}{NW1vYGPv-29jjeg-7}\nwenddeflinemarkup
RESULTS_FAILED?=      -\\|Fx\\?\\|U
RESULTS_FAILED_regex= "\\( \\|  \\|,\\)\\"\\?\\($\{RESULTS_FAILED\}\\)\\"\\?\\(  .*\\)*$$"
\nwused{\\{NW1vYGPv-4UwcQh-1}}\nwendcode{}\nwbegindocs{32}The code to filter the data will thus be:
\nwenddocs{}\nwbegincode{33}\sublabel{NW1vYGPv-257BvH-2}\nwmargintag{{\nwtagstyle{}\subpageref{NW1vYGPv-257BvH-2}}}\moddef{diff recipe~{\nwtagstyle{}\subpageref{NW1vYGPv-257BvH-1}}}\plusendmoddef\nwstartdeflinemarkup\nwusesondefline{\\{NW1vYGPv-2InVV3-4}}\nwprevnextdefs{NW1vYGPv-257BvH-1}{NW1vYGPv-257BvH-3}\nwenddeflinemarkup
$(if $\{RESULTS_FAILED\},$\{GREP\} -v $\{RESULTS_FAILED_regex\} |,) \\
\nwused{\\{NW1vYGPv-2InVV3-4}}\nwendcode{}\nwbegindocs{34}\nwdocspar

Finally, we want to compare this result with the old result and keep only the 
modified lines.
For the convenience of the secretaries we will also sort the data on the third 
column (\ie family name).
\nwenddocs{}\nwbegincode{35}\sublabel{NW1vYGPv-257BvH-3}\nwmargintag{{\nwtagstyle{}\subpageref{NW1vYGPv-257BvH-3}}}\moddef{diff recipe~{\nwtagstyle{}\subpageref{NW1vYGPv-257BvH-1}}}\plusendmoddef\nwstartdeflinemarkup\nwusesondefline{\\{NW1vYGPv-2InVV3-4}}\nwprevnextdefs{NW1vYGPv-257BvH-2}{\relax}\nwenddeflinemarkup
$\{DIFF\} $\{@:.diff=\} - | $\{SED\} -n "/^> /s/^> //p" | $\{SORT\} -k 3 > $@
\nwused{\\{NW1vYGPv-2InVV3-4}}\nwendcode{}\nwbegindocs{36}Due to the difference comparison we require that the new file depends on the 
old file, and that the old file actually exists, so we add this as 
a dependency.
\nwenddocs{}\nwbegincode{37}\sublabel{NW1vYGPv-2InVV3-5}\nwmargintag{{\nwtagstyle{}\subpageref{NW1vYGPv-2InVV3-5}}}\moddef{targets~{\nwtagstyle{}\subpageref{NW1vYGPv-2InVV3-1}}}\plusendmoddef\nwstartdeflinemarkup\nwusesondefline{\\{NW1vYGPv-4UwcQh-1}}\nwprevnextdefs{NW1vYGPv-2InVV3-4}{NW1vYGPv-2InVV3-6}\nwenddeflinemarkup
$\{out\}.diff: $\{out\}
\nwused{\\{NW1vYGPv-4UwcQh-1}}\nwendcode{}\nwbegindocs{38}We will have to ensure that the old file exists, \ie create it if it does 
not exist.
We need it for comparisons, an empty file perfectly represents previously
non-existent results, so we create a symbolic link to {\Tt{}/dev/null\nwendquote}.
\nwenddocs{}\nwbegincode{39}\sublabel{NW1vYGPv-2InVV3-6}\nwmargintag{{\nwtagstyle{}\subpageref{NW1vYGPv-2InVV3-6}}}\moddef{targets~{\nwtagstyle{}\subpageref{NW1vYGPv-2InVV3-1}}}\plusendmoddef\nwstartdeflinemarkup\nwusesondefline{\\{NW1vYGPv-4UwcQh-1}}\nwprevnextdefs{NW1vYGPv-2InVV3-5}{NW1vYGPv-2InVV3-7}\nwenddeflinemarkup
$\{out\}:
  [ -r $@ ] || $\{LN\} -s /dev/null $@
\nwused{\\{NW1vYGPv-4UwcQh-1}}\nwendcode{}\nwbegindocs{40}\nwdocspar

\subsection{Extracting identifiers for reporting}
\label{ChangingIdentifiers}

In Moodle, every student is identified by a unique username.
In the national database of grades, every student is uniquely identified by 
their civic registration number.
What we want to do here is to extract the username from the exported data, then
supply the corresponding civic registration numbers.
We are only interested in the usernames of those students for whom the results 
changed, so we can use the {\Tt{}{\$}{\nwlbrace}out{\nwrbrace}.diff\nwendquote} file from above.

In this case we can use one of make's suffix-based constructions.
We will take a file with suffix \enquote{.csv.diff} and create a file 
\enquote{.csv.diff.id}.
\nwenddocs{}\nwbegincode{41}\sublabel{NW1vYGPv-2InVV3-7}\nwmargintag{{\nwtagstyle{}\subpageref{NW1vYGPv-2InVV3-7}}}\moddef{targets~{\nwtagstyle{}\subpageref{NW1vYGPv-2InVV3-1}}}\plusendmoddef\nwstartdeflinemarkup\nwusesondefline{\\{NW1vYGPv-4UwcQh-1}}\nwprevnextdefs{NW1vYGPv-2InVV3-6}{NW1vYGPv-2InVV3-8}\nwenddeflinemarkup
.SUFFIXES: .csv .csv.diff .csv.diff.id
.csv.diff.csv.diff.id:
  \LA{}identifier recipe~{\nwtagstyle{}\subpageref{NW1vYGPv-3Cm36S-1}}\RA{}
\nwused{\\{NW1vYGPv-4UwcQh-1}}\nwendcode{}\nwbegindocs{42}Now we want to extract the usernames and get the identifiers from the 
national database.
We simply extract the list of usernames (the third column in the data) and pipe
it to a pager.
\nwenddocs{}\nwbegincode{43}\sublabel{NW1vYGPv-3Cm36S-1}\nwmargintag{{\nwtagstyle{}\subpageref{NW1vYGPv-3Cm36S-1}}}\moddef{identifier recipe~{\nwtagstyle{}\subpageref{NW1vYGPv-3Cm36S-1}}}\endmoddef\nwstartdeflinemarkup\nwusesondefline{\\{NW1vYGPv-2InVV3-7}}\nwprevnextdefs{\relax}{NW1vYGPv-3Cm36S-2}\nwenddeflinemarkup
@echo "---- userids showed in $\{PAGER\} ----"
$\{CAT\} $< | $\{CUT\} -f 3 | $\{PAGER\}
\nwalsodefined{\\{NW1vYGPv-3Cm36S-2}}\nwused{\\{NW1vYGPv-2InVV3-7}}\nwendcode{}\nwbegindocs{44}Now we let the user paste the list of both identifiers.
\nwenddocs{}\nwbegincode{45}\sublabel{NW1vYGPv-3Cm36S-2}\nwmargintag{{\nwtagstyle{}\subpageref{NW1vYGPv-3Cm36S-2}}}\moddef{identifier recipe~{\nwtagstyle{}\subpageref{NW1vYGPv-3Cm36S-1}}}\plusendmoddef\nwstartdeflinemarkup\nwusesondefline{\\{NW1vYGPv-2InVV3-7}}\nwprevnextdefs{NW1vYGPv-3Cm36S-1}{\relax}\nwenddeflinemarkup
@echo "---- paste username <tab> personnummer, end with C-d on a blank line (EOF) ----"
$\{CAT\} > $@
\nwused{\\{NW1vYGPv-2InVV3-7}}\nwendcode{}\nwbegindocs{46}\nwdocspar

\subsection{Generating the report}
\label{GeneratingReport}

Now we have the changes in {\Tt{}{\$}{\nwlbrace}out{\nwrbrace}.diff\nwendquote} and a mapping from usernames to 
civic registration numbers in {\Tt{}{\$}{\nwlbrace}out{\nwrbrace}.diff.id\nwendquote}.
To create the report, we only have to join these files and convert the result 
to PDF format.

To convert a CSV-file to PDF we will use LibreOffice and one of make's suffix 
rules.
\nwenddocs{}\nwbegincode{47}\sublabel{NW1vYGPv-2InVV3-8}\nwmargintag{{\nwtagstyle{}\subpageref{NW1vYGPv-2InVV3-8}}}\moddef{targets~{\nwtagstyle{}\subpageref{NW1vYGPv-2InVV3-1}}}\plusendmoddef\nwstartdeflinemarkup\nwusesondefline{\\{NW1vYGPv-4UwcQh-1}}\nwprevnextdefs{NW1vYGPv-2InVV3-7}{NW1vYGPv-2InVV3-9}\nwenddeflinemarkup
.SUFFIXES: .csv .pdf
.csv.pdf:
  $\{LOCALC\} $<
\nwused{\\{NW1vYGPv-4UwcQh-1}}\nwendcode{}\nwbegindocs{48}Now we can add a target using this conversion.
\nwenddocs{}\nwbegincode{49}\sublabel{NW1vYGPv-2InVV3-9}\nwmargintag{{\nwtagstyle{}\subpageref{NW1vYGPv-2InVV3-9}}}\moddef{targets~{\nwtagstyle{}\subpageref{NW1vYGPv-2InVV3-1}}}\plusendmoddef\nwstartdeflinemarkup\nwusesondefline{\\{NW1vYGPv-4UwcQh-1}}\nwprevnextdefs{NW1vYGPv-2InVV3-8}{NW1vYGPv-2InVV3-A}\nwenddeflinemarkup
$\{report:.csv=.pdf\}: $\{report:.pdf=.csv\}
\nwused{\\{NW1vYGPv-4UwcQh-1}}\nwendcode{}\nwbegindocs{50}\nwdocspar

The above target lets us create a PDF-formatted report from a CSV-file, so now 
we have to create that CSV-file using {\Tt{}{\$}{\nwlbrace}out{\nwrbrace}.diff\nwendquote} and {\Tt{}{\$}{\nwlbrace}out{\nwrbrace}.diff.id\nwendquote}.
We also need the table headers from {\Tt{}{\$}{\nwlbrace}in{\nwrbrace}\nwendquote}.
\nwenddocs{}\nwbegincode{51}\sublabel{NW1vYGPv-2InVV3-A}\nwmargintag{{\nwtagstyle{}\subpageref{NW1vYGPv-2InVV3-A}}}\moddef{targets~{\nwtagstyle{}\subpageref{NW1vYGPv-2InVV3-1}}}\plusendmoddef\nwstartdeflinemarkup\nwusesondefline{\\{NW1vYGPv-4UwcQh-1}}\nwprevnextdefs{NW1vYGPv-2InVV3-9}{NW1vYGPv-2InVV3-B}\nwenddeflinemarkup
$\{report:.pdf=.csv\}: $\{in\} $\{out\}.diff $\{out\}.diff.id
  \LA{}report.csv recipe~{\nwtagstyle{}\subpageref{NW1vYGPv-1PqcL7-1}}\RA{}
\nwused{\\{NW1vYGPv-4UwcQh-1}}\nwendcode{}\nwbegindocs{52}Since the target for {\Tt{}report.csv\nwendquote} will automatically generate 
{\Tt{}{\$}{\nwlbrace}out{\nwrbrace}.diff\nwendquote} and {\Tt{}{\$}{\nwlbrace}out{\nwrbrace}.diff.id\nwendquote} we would better add them to the clean 
recipe in addition to the {\Tt{}{\$}{\nwlbrace}report:.pdf=.csv{\nwrbrace}\nwendquote} file.
\nwenddocs{}\nwbegincode{53}\sublabel{NW1vYGPv-mtljB-2}\nwmargintag{{\nwtagstyle{}\subpageref{NW1vYGPv-mtljB-2}}}\moddef{clean recipe~{\nwtagstyle{}\subpageref{NW1vYGPv-mtljB-1}}}\plusendmoddef\nwstartdeflinemarkup\nwusesondefline{\\{NW1vYGPv-2InVV3-2}}\nwprevnextdefs{NW1vYGPv-mtljB-1}{NW1vYGPv-mtljB-3}\nwenddeflinemarkup
$\{RM\} $\{out\}.diff $\{out\}.diff.id
$\{RM\} $\{report:.pdf=.csv\}
\nwused{\\{NW1vYGPv-2InVV3-2}}\nwendcode{}\nwbegindocs{54}\nwdocspar

Now we want the header back, so we can get it properly formatted from 
{\Tt{}{\$}{\nwlbrace}out{\nwrbrace}.new\nwendquote}.
However, we do not want all the excess columns for the grades: we only need one
column with a summary.
\nwenddocs{}\nwbegincode{55}\sublabel{NW1vYGPv-1PqcL7-1}\nwmargintag{{\nwtagstyle{}\subpageref{NW1vYGPv-1PqcL7-1}}}\moddef{report.csv recipe~{\nwtagstyle{}\subpageref{NW1vYGPv-1PqcL7-1}}}\endmoddef\nwstartdeflinemarkup\nwusesondefline{\\{NW1vYGPv-2InVV3-A}}\nwprevnextdefs{\relax}{NW1vYGPv-1PqcL7-2}\nwenddeflinemarkup
$\{HEAD\} -n 1 $\{out\}.new | \\
  $\{CUT\} -f -$\{RESULTS_COLUMNS\} > $@
\nwalsodefined{\\{NW1vYGPv-1PqcL7-2}}\nwused{\\{NW1vYGPv-2InVV3-A}}\nwendcode{}\nwbegindocs{56}Next, we simply join the two tables on the username column and sort the list 
on the column of the family name.
We want to cut the excess columns here as well, the number of columns is 
controlled by {\Tt{}{\$}{\nwlbrace}RESULTS{\_}COLUMNS{\nwrbrace}\nwendquote}.
\nwenddocs{}\nwbegincode{57}\sublabel{NW1vYGPv-1PqcL7-2}\nwmargintag{{\nwtagstyle{}\subpageref{NW1vYGPv-1PqcL7-2}}}\moddef{report.csv recipe~{\nwtagstyle{}\subpageref{NW1vYGPv-1PqcL7-1}}}\plusendmoddef\nwstartdeflinemarkup\nwusesondefline{\\{NW1vYGPv-2InVV3-A}}\nwprevnextdefs{NW1vYGPv-1PqcL7-1}{\relax}\nwenddeflinemarkup
$\{JOIN\} -1 1 -2 3 $\{out\}.diff.id $\{out\}.diff | $\{CUT\} -d " " -f 2- | \\
  $\{SORT\} -k 2 | $\{CUT\} -d " " -f -$\{RESULTS_COLUMNS\} >> $@
\nwused{\\{NW1vYGPv-2InVV3-A}}\nwendcode{}\nwbegindocs{58}By default we let the default number of columns be four, \ie first and last 
name, civic identification number and finally one grade.
\nwenddocs{}\nwbegincode{59}\sublabel{NW1vYGPv-29jjeg-7}\nwmargintag{{\nwtagstyle{}\subpageref{NW1vYGPv-29jjeg-7}}}\moddef{variables~{\nwtagstyle{}\subpageref{NW1vYGPv-29jjeg-1}}}\plusendmoddef\nwstartdeflinemarkup\nwusesondefline{\\{NW1vYGPv-4UwcQh-1}}\nwprevnextdefs{NW1vYGPv-29jjeg-6}{\relax}\nwenddeflinemarkup
RESULTS_COLUMNS?=   4
\nwused{\\{NW1vYGPv-4UwcQh-1}}\nwendcode{}\nwbegindocs{60}\nwdocspar


\section{Sending and storing the results}

Now we will do the actual reporting.
As stated in \cref{Intro} we have a target {\Tt{}report\nwendquote} for this purpose.
\nwenddocs{}\nwbegincode{61}\sublabel{NW1vYGPv-2InVV3-B}\nwmargintag{{\nwtagstyle{}\subpageref{NW1vYGPv-2InVV3-B}}}\moddef{targets~{\nwtagstyle{}\subpageref{NW1vYGPv-2InVV3-1}}}\plusendmoddef\nwstartdeflinemarkup\nwusesondefline{\\{NW1vYGPv-4UwcQh-1}}\nwprevnextdefs{NW1vYGPv-2InVV3-A}{\relax}\nwenddeflinemarkup
.PHONY: report
report: $\{report\} $\{in\}
\nwused{\\{NW1vYGPv-4UwcQh-1}}\nwendcode{}\nwbegindocs{62}Since this will trigger the creation of {\Tt{}{\$}{\nwlbrace}report{\nwrbrace}\nwendquote} we must add it to the 
clean recipe.
\nwenddocs{}\nwbegincode{63}\sublabel{NW1vYGPv-mtljB-3}\nwmargintag{{\nwtagstyle{}\subpageref{NW1vYGPv-mtljB-3}}}\moddef{clean recipe~{\nwtagstyle{}\subpageref{NW1vYGPv-mtljB-1}}}\plusendmoddef\nwstartdeflinemarkup\nwusesondefline{\\{NW1vYGPv-2InVV3-2}}\nwprevnextdefs{NW1vYGPv-mtljB-2}{\relax}\nwenddeflinemarkup
$\{RM\} $\{report\}
\nwused{\\{NW1vYGPv-2InVV3-2}}\nwendcode{}\nwbegindocs{64}\nwdocspar

If there are no new results, then we do not want to send any report.
The first thing we do is thus to check for new results.
If there are none, we will say so to the user.
\nwenddocs{}\nwbegincode{65}\sublabel{NW1vYGPv-1bP5sY-1}\nwmargintag{{\nwtagstyle{}\subpageref{NW1vYGPv-1bP5sY-1}}}\moddef{report recipe~{\nwtagstyle{}\subpageref{NW1vYGPv-1bP5sY-1}}}\endmoddef\nwstartdeflinemarkup\nwusesondefline{\\{NW1vYGPv-2InVV3-1}}\nwprevnextdefs{\relax}{NW1vYGPv-1bP5sY-2}\nwenddeflinemarkup
if [ ! -s $\{out\}.diff ]; then \\
  echo "No new results to report" >&2; \\
\nwalsodefined{\\{NW1vYGPv-1bP5sY-2}}\nwused{\\{NW1vYGPv-2InVV3-1}}\nwendcode{}\nwbegindocs{66}Otherwise, if there are new results, we will output them using the pager and 
then send them using the email program.
If the emailing succeeds, then we want to store the results, but not if it 
fails (hence the conditional).
\nwenddocs{}\nwbegincode{67}\sublabel{NW1vYGPv-1bP5sY-2}\nwmargintag{{\nwtagstyle{}\subpageref{NW1vYGPv-1bP5sY-2}}}\moddef{report recipe~{\nwtagstyle{}\subpageref{NW1vYGPv-1bP5sY-1}}}\plusendmoddef\nwstartdeflinemarkup\nwusesondefline{\\{NW1vYGPv-2InVV3-1}}\nwprevnextdefs{NW1vYGPv-1bP5sY-1}{\relax}\nwenddeflinemarkup
else \\
  $\{PAGER\} $\{report\}; \\
  $\{RESULTS_MAILER\} && \\
  $\{MV\} $\{out\}.new $\{out\}; \\
fi
\nwused{\\{NW1vYGPv-2InVV3-1}}\nwendcode{}

\nwixlogsorted{c}{{clean recipe}{NW1vYGPv-mtljB-1}{\nwixu{NW1vYGPv-2InVV3-2}\nwixd{NW1vYGPv-mtljB-1}\nwixd{NW1vYGPv-mtljB-2}\nwixd{NW1vYGPv-mtljB-3}}}%
\nwixlogsorted{c}{{diff recipe}{NW1vYGPv-257BvH-1}{\nwixu{NW1vYGPv-2InVV3-4}\nwixd{NW1vYGPv-257BvH-1}\nwixd{NW1vYGPv-257BvH-2}\nwixd{NW1vYGPv-257BvH-3}}}%
\nwixlogsorted{c}{{identifier recipe}{NW1vYGPv-3Cm36S-1}{\nwixu{NW1vYGPv-2InVV3-7}\nwixd{NW1vYGPv-3Cm36S-1}\nwixd{NW1vYGPv-3Cm36S-2}}}%
\nwixlogsorted{c}{{new recipe}{NW1vYGPv-HUskc-1}{\nwixu{NW1vYGPv-2InVV3-3}\nwixd{NW1vYGPv-HUskc-1}\nwixd{NW1vYGPv-HUskc-2}\nwixd{NW1vYGPv-HUskc-3}}}%
\nwixlogsorted{c}{{report recipe}{NW1vYGPv-1bP5sY-1}{\nwixu{NW1vYGPv-2InVV3-1}\nwixd{NW1vYGPv-1bP5sY-1}\nwixd{NW1vYGPv-1bP5sY-2}}}%
\nwixlogsorted{c}{{report.csv recipe}{NW1vYGPv-1PqcL7-1}{\nwixu{NW1vYGPv-2InVV3-A}\nwixd{NW1vYGPv-1PqcL7-1}\nwixd{NW1vYGPv-1PqcL7-2}}}%
\nwixlogsorted{c}{{results.mk}{NW1vYGPv-4UwcQh-1}{\nwixd{NW1vYGPv-4UwcQh-1}}}%
\nwixlogsorted{c}{{targets}{NW1vYGPv-2InVV3-1}{\nwixu{NW1vYGPv-4UwcQh-1}\nwixd{NW1vYGPv-2InVV3-1}\nwixd{NW1vYGPv-2InVV3-2}\nwixd{NW1vYGPv-2InVV3-3}\nwixd{NW1vYGPv-2InVV3-4}\nwixd{NW1vYGPv-2InVV3-5}\nwixd{NW1vYGPv-2InVV3-6}\nwixd{NW1vYGPv-2InVV3-7}\nwixd{NW1vYGPv-2InVV3-8}\nwixd{NW1vYGPv-2InVV3-9}\nwixd{NW1vYGPv-2InVV3-A}\nwixd{NW1vYGPv-2InVV3-B}}}%
\nwixlogsorted{c}{{variables}{NW1vYGPv-29jjeg-1}{\nwixd{NW1vYGPv-29jjeg-1}\nwixu{NW1vYGPv-4UwcQh-1}\nwixd{NW1vYGPv-29jjeg-2}\nwixd{NW1vYGPv-29jjeg-3}\nwixd{NW1vYGPv-29jjeg-4}\nwixd{NW1vYGPv-29jjeg-5}\nwixd{NW1vYGPv-29jjeg-6}\nwixd{NW1vYGPv-29jjeg-7}}}%
\nwbegindocs{68}\nwdocspar
\nwenddocs{}
