\section{Introduction and usage}% ===> this file was generated automatically by noweave --- better not edit it
\label{tex:Intro}

The aim of this include file is to make building LaTeX documents easier.
First we want to add suffix rules for LaTeX similar to those already in 
make(1)~\cite[see][Sect.\ 10.2]{GNUMake} for \eg (plain) TeX and C.

The main goal is to improve the default rules of GNU make; to change from plain 
TeX to LaTeX, to change from DVI-format to PDF-format.
We provide several suffix rules:
first, for ordinary documents, \ie to compile a {\Tt{}.tex\nwendquote} file to {\Tt{}.pdf\nwendquote},
but we also provide suffix rules for {\Tt{}.dvi\nwendquote};
second, for classes and packages, \ie to compile
\begin{enumerate*}
\item a DocTeX {\Tt{}.dtx\nwendquote} file to {\Tt{}.pdf\nwendquote} or {\Tt{}.dvi\nwendquote} and
\item an {\Tt{}.ins\nwendquote} file to {\Tt{}.cls\nwendquote} or {\Tt{}.sty\nwendquote}.
\end{enumerate*}
The suffix rules we provide here follows the conventions set out 
in~\cite[Sect.\ 10.2]{GNUMake}.

To do the main LaTeX compilation, we use the following set up:
\nwfilename{tex.mk.nw}\nwbegincode{1}\sublabel{NW4GQOIU-29jjeg-1}\nwmargintag{{\nwtagstyle{}\subpageref{NW4GQOIU-29jjeg-1}}}\moddef{variables~{\nwtagstyle{}\subpageref{NW4GQOIU-29jjeg-1}}}\endmoddef\nwstartdeflinemarkup\nwusesondefline{\\{NW4GQOIU-2kaaDB-1}}\nwprevnextdefs{\relax}{NW4GQOIU-29jjeg-2}\nwenddeflinemarkup
LATEX?=           latexmk -dvi -use-make
PDFLATEX?=        latexmk -pdf -use-make
LATEXFLAGS?=
PREPROCESS.tex?=  $\{PDFLATEX\} $\{LATEXFLAGS\} $<
PREPROCESS.dtx?=  $\{PREPROCESS.tex\}
TEX_OUTDIR?=      ltxobj
COMPILE.tex?=     \\
  $\{PDFLATEX\} $\{LATEXFLAGS\} -output-directory=$\{TEX_OUTDIR\} $<; \\
  while (grep "Rerun to get cross" $\{TEX_OUTDIR\}/$\{<:.tex=.log\}); \\
    do $\{PDFLATEX\} $\{LATEXFLAGS\} -output-directory=$\{TEX_OUTDIR\} $<; \\
  done
COMPILE.dtx?=     $\{COMPILE.tex\}
\nwalsodefined{\\{NW4GQOIU-29jjeg-2}\\{NW4GQOIU-29jjeg-3}\\{NW4GQOIU-29jjeg-4}\\{NW4GQOIU-29jjeg-5}\\{NW4GQOIU-29jjeg-6}\\{NW4GQOIU-29jjeg-7}\\{NW4GQOIU-29jjeg-8}\\{NW4GQOIU-29jjeg-9}}\nwused{\\{NW4GQOIU-2kaaDB-1}}\nwendcode{}\nwbegindocs{2}Thus one would have to change the {\Tt{}COMPILE.tex\nwendquote} variable to change format 
from PDF to DVI unless a suffix rule is used.

%We note that we add this output directory to the search path for 
%prerequisites~\cite[see][Sect.\ 4.5.1]{GNUMake}.
%The reason for this is that we might need some out the objects as prerequisites 
%for other files.

Normally, the above is all that is needed.
However, if you need to manually build the bibliography, you can either add the
{\Tt{}.bbl\nwendquote} file as a prerequisite or set the following variable to a non-empty 
string.
\nwenddocs{}\nwbegincode{3}\sublabel{NW4GQOIU-29jjeg-2}\nwmargintag{{\nwtagstyle{}\subpageref{NW4GQOIU-29jjeg-2}}}\moddef{variables~{\nwtagstyle{}\subpageref{NW4GQOIU-29jjeg-1}}}\plusendmoddef\nwstartdeflinemarkup\nwusesondefline{\\{NW4GQOIU-2kaaDB-1}}\nwprevnextdefs{NW4GQOIU-29jjeg-1}{NW4GQOIU-29jjeg-3}\nwenddeflinemarkup
TEX_BBL?=
\nwused{\\{NW4GQOIU-2kaaDB-1}}\nwendcode{}\nwbegindocs{4}Similarly as for the main LaTeX commands, the bibtex(1) command is controlled 
by
\nwenddocs{}\nwbegincode{5}\sublabel{NW4GQOIU-29jjeg-3}\nwmargintag{{\nwtagstyle{}\subpageref{NW4GQOIU-29jjeg-3}}}\moddef{variables~{\nwtagstyle{}\subpageref{NW4GQOIU-29jjeg-1}}}\plusendmoddef\nwstartdeflinemarkup\nwusesondefline{\\{NW4GQOIU-2kaaDB-1}}\nwprevnextdefs{NW4GQOIU-29jjeg-2}{NW4GQOIU-29jjeg-4}\nwenddeflinemarkup
BIBTEX?=            bibtexu
BIBTEXFLAGS?=
BIBLIOGRAPHY.aux?=  $\{BIBTEX\} $\{BIBTEXFLAGS\} $<
\nwused{\\{NW4GQOIU-2kaaDB-1}}\nwendcode{}\nwbegindocs{6}And, in case we use biblatex, the biber(1) command is controlled by
\nwenddocs{}\nwbegincode{7}\sublabel{NW4GQOIU-29jjeg-4}\nwmargintag{{\nwtagstyle{}\subpageref{NW4GQOIU-29jjeg-4}}}\moddef{variables~{\nwtagstyle{}\subpageref{NW4GQOIU-29jjeg-1}}}\plusendmoddef\nwstartdeflinemarkup\nwusesondefline{\\{NW4GQOIU-2kaaDB-1}}\nwprevnextdefs{NW4GQOIU-29jjeg-3}{NW4GQOIU-29jjeg-5}\nwenddeflinemarkup
BIBER?=             biber
BIBERFLAGS?=
BIBLIOGRAPHY.bcf?=  $\{BIBER\} -O $@ $\{BIBERFLAGS\} $<
\nwused{\\{NW4GQOIU-2kaaDB-1}}\nwendcode{}\nwbegindocs{8}\nwdocspar

Similarly as for the bibliography, to enable indexing you can either manually 
add the {\Tt{}.ind\nwendquote} file as a prerequisite, or you can set the following variable 
to a non-empty string.
\nwenddocs{}\nwbegincode{9}\sublabel{NW4GQOIU-29jjeg-5}\nwmargintag{{\nwtagstyle{}\subpageref{NW4GQOIU-29jjeg-5}}}\moddef{variables~{\nwtagstyle{}\subpageref{NW4GQOIU-29jjeg-1}}}\plusendmoddef\nwstartdeflinemarkup\nwusesondefline{\\{NW4GQOIU-2kaaDB-1}}\nwprevnextdefs{NW4GQOIU-29jjeg-4}{NW4GQOIU-29jjeg-6}\nwenddeflinemarkup
TEX_IND?=
\nwused{\\{NW4GQOIU-2kaaDB-1}}\nwendcode{}\nwbegindocs{10}The indexing-related programs are the following.
\nwenddocs{}\nwbegincode{11}\sublabel{NW4GQOIU-29jjeg-6}\nwmargintag{{\nwtagstyle{}\subpageref{NW4GQOIU-29jjeg-6}}}\moddef{variables~{\nwtagstyle{}\subpageref{NW4GQOIU-29jjeg-1}}}\plusendmoddef\nwstartdeflinemarkup\nwusesondefline{\\{NW4GQOIU-2kaaDB-1}}\nwprevnextdefs{NW4GQOIU-29jjeg-5}{NW4GQOIU-29jjeg-7}\nwenddeflinemarkup
XINDY?=       texindy
XINDYFLAGS?=
COMPILE.idx?= $\{XINDY\} $\{OUTPUT_OPTION\} $\{XINDYFLAGS\} $<

MAKEINDEX?=   makeindex
MAKEIDXFLAGS?=
COMPILE.nlo?= $\{MAKEINDEX\} $\{OUTPUT_OPTION\} $\{MAKEIDXFLAGS\} -s nomencl.ist $<
\nwused{\\{NW4GQOIU-2kaaDB-1}}\nwendcode{}\nwbegindocs{12}\nwdocspar

We also provide support for PythonTeX.
This is enabled by the following variable.
\nwenddocs{}\nwbegincode{13}\sublabel{NW4GQOIU-29jjeg-7}\nwmargintag{{\nwtagstyle{}\subpageref{NW4GQOIU-29jjeg-7}}}\moddef{variables~{\nwtagstyle{}\subpageref{NW4GQOIU-29jjeg-1}}}\plusendmoddef\nwstartdeflinemarkup\nwusesondefline{\\{NW4GQOIU-2kaaDB-1}}\nwprevnextdefs{NW4GQOIU-29jjeg-6}{NW4GQOIU-29jjeg-8}\nwenddeflinemarkup
TEX_PYTHONTEX?=
\nwused{\\{NW4GQOIU-2kaaDB-1}}\nwendcode{}\nwbegindocs{14}Then the required command and flags are controlled with the following 
variables.
\nwenddocs{}\nwbegincode{15}\sublabel{NW4GQOIU-29jjeg-8}\nwmargintag{{\nwtagstyle{}\subpageref{NW4GQOIU-29jjeg-8}}}\moddef{variables~{\nwtagstyle{}\subpageref{NW4GQOIU-29jjeg-1}}}\plusendmoddef\nwstartdeflinemarkup\nwusesondefline{\\{NW4GQOIU-2kaaDB-1}}\nwprevnextdefs{NW4GQOIU-29jjeg-7}{NW4GQOIU-29jjeg-9}\nwenddeflinemarkup
PYTHONTEX?=   pythontex
PYTHONTEXFLAGS?=
\nwused{\\{NW4GQOIU-2kaaDB-1}}\nwendcode{}\nwbegindocs{16}\nwdocspar

Finally, we provide targets to easily add external classes as dependencies.
We add the phony targets
\begin{itemize}
\item{\Tt{}lncs\nwendquote} for Springer \ac{LNCS},
\item{\Tt{}biblatex-lncs\nwendquote} for the \ac{LNCS} bibliography style for the 
{\Tt{}biblatex\nwendquote} package,
\item{\Tt{}rfc\nwendquote} or {\Tt{}rfc.bib\nwendquote} for an up-to-date bibliography containing all 
\ac{IETF} \acp{RFC},
\item{\Tt{}popets\nwendquote} for the style files of the Proceedings of the Privacy Enhancing 
Technologies (PoPETS, De Gruyter Open/Sciendo).
\end{itemize}

Finally, we add a method to quickly merge bibliographies without redundancy.
This is based on the archive syntax of make(1).
This allows us to have the resulting full bibliography {\Tt{}full.bib\nwendquote} consist of 
the subbibliographies {\Tt{}sub0.bib\nwendquote} and {\Tt{}sub1.bib\nwendquote}:
\begin{lstlisting}
full.bib: full.bib(sub0.bib sub1.bib)
\end{lstlisting}
Or, directly as a dependency for the PDF:
\begin{lstlisting}
paper.pdf: full.bib(sub0.bib sub1.bib)
\end{lstlisting}


\section{Implementation overview}

The structure of the include file is similar to a header file in C or C++.
The include file uses the old C-style technique to prevent multiple inclusions.
\nwenddocs{}\nwbegincode{17}\sublabel{NW4GQOIU-2kaaDB-1}\nwmargintag{{\nwtagstyle{}\subpageref{NW4GQOIU-2kaaDB-1}}}\moddef{tex.mk~{\nwtagstyle{}\subpageref{NW4GQOIU-2kaaDB-1}}}\endmoddef\nwstartdeflinemarkup\nwenddeflinemarkup
ifndef TEX_MK
TEX_MK=true

.NOTPARALLEL:

INCLUDE_MAKEFILES?=.
include $\{INCLUDE_MAKEFILES\}/portability.mk

\LA{}variables~{\nwtagstyle{}\subpageref{NW4GQOIU-29jjeg-1}}\RA{}
\LA{}targets for documents~{\nwtagstyle{}\subpageref{NW4GQOIU-4QwfwR-1}}\RA{}
\LA{}targets for class and package files~{\nwtagstyle{}\subpageref{NW4GQOIU-tejPg-1}}\RA{}
\LA{}targets for external classes~{\nwtagstyle{}\subpageref{NW4GQOIU-2Oad4t-1}}\RA{}
\LA{}targets for cleaning~{\nwtagstyle{}\subpageref{NW4GQOIU-LrkAl-1}}\RA{}

endif
\nwnotused{tex.mk}\nwendcode{}\nwbegindocs{18}We include {\Tt{}portability.mk\nwendquote} (\cref{portability.mk}) to get portable 
settings for several common utilities.
It is currently not possible to run the TeX-builds in parallel, hence the need 
for the {\Tt{}.NOTPARALLEL\nwendquote} target --- this runs any jobs that includes {\Tt{}tex.mk\nwendquote} 
  in serial.
However, this is not inherited, any recursive make invocations will run in 
parallel unless {\Tt{}.NOTPARALLEL\nwendquote} is specified there as well.

We will start with the targets for cleaning.
We provide two phony targets, {\Tt{}clean-tex\nwendquote} and {\Tt{}distclean-tex\nwendquote}, and we 
add them as prerequisites to {\Tt{}clean\nwendquote} and {\Tt{}distclean\nwendquote}, respectively.
\nwenddocs{}\nwbegincode{19}\sublabel{NW4GQOIU-LrkAl-1}\nwmargintag{{\nwtagstyle{}\subpageref{NW4GQOIU-LrkAl-1}}}\moddef{targets for cleaning~{\nwtagstyle{}\subpageref{NW4GQOIU-LrkAl-1}}}\endmoddef\nwstartdeflinemarkup\nwusesondefline{\\{NW4GQOIU-2kaaDB-1}}\nwenddeflinemarkup
.PHONY: clean clean-tex
clean: clean-tex

clean-tex:
  \LA{}clean recipe~{\nwtagstyle{}\subpageref{NW4GQOIU-mtljB-1}}\RA{}

.PHONY: distclean distclean-tex
distclean: distclean-tex

distclean-tex:
  \LA{}distclean recipe~{\nwtagstyle{}\subpageref{NW4GQOIU-1Hidyl-1}}\RA{}
\nwused{\\{NW4GQOIU-2kaaDB-1}}\nwendcode{}\nwbegindocs{20}We will add to the recipes in the remainder of the chapter.
However, as latexmk(1) is set as the default in \cref{tex:Intro}, we can 
already add the following line to the cleaning recipe.
\nwenddocs{}\nwbegincode{21}\sublabel{NW4GQOIU-mtljB-1}\nwmargintag{{\nwtagstyle{}\subpageref{NW4GQOIU-mtljB-1}}}\moddef{clean recipe~{\nwtagstyle{}\subpageref{NW4GQOIU-mtljB-1}}}\endmoddef\nwstartdeflinemarkup\nwusesondefline{\\{NW4GQOIU-LrkAl-1}}\nwenddeflinemarkup
-latexmk -C -output-directory=$\{TEX_OUTDIR\}
[ "$\{TEX_OUTDIR\}" -ef "$$(pwd)" ] || \\
  $\{RM\} -R $\{TEX_OUTDIR\}
$\{RM\} *.pytxcode
$\{RM\} -R pythontex-files-*
\nwused{\\{NW4GQOIU-LrkAl-1}}\nwendcode{}\nwbegindocs{22}\nwdocspar


\section{Targets for documents}

Now we will treat how to compile documents.
\nwenddocs{}\nwbegincode{23}\sublabel{NW4GQOIU-4QwfwR-1}\nwmargintag{{\nwtagstyle{}\subpageref{NW4GQOIU-4QwfwR-1}}}\moddef{targets for documents~{\nwtagstyle{}\subpageref{NW4GQOIU-4QwfwR-1}}}\endmoddef\nwstartdeflinemarkup\nwusesondefline{\\{NW4GQOIU-2kaaDB-1}}\nwenddeflinemarkup
\LA{}auxillary files~{\nwtagstyle{}\subpageref{NW4GQOIU-SIrPU-1}}\RA{}
\LA{}bibliography files~{\nwtagstyle{}\subpageref{NW4GQOIU-3g0BAK-1}}\RA{}
\LA{}indices files~{\nwtagstyle{}\subpageref{NW4GQOIU-UTABh-1}}\RA{}
\LA{}PythonTeX files~{\nwtagstyle{}\subpageref{NW4GQOIU-287s1n-1}}\RA{}
\LA{}document files~{\nwtagstyle{}\subpageref{NW4GQOIU-2cBbLf-1}}\RA{}
\LA{}target for latexmkrc~{\nwtagstyle{}\subpageref{NW4GQOIU-4Jzoxc-1}}\RA{}
\nwused{\\{NW4GQOIU-2kaaDB-1}}\nwendcode{}\nwbegindocs{24}These will be discussed in the following sections.
However, since we use latexmk(1) by default (\cref{tex:Intro}), we will discuss 
the relevant {\Tt{}\LA{}latexmkrc~{\nwtagstyle{}\subpageref{NW4GQOIU-3qE2M6-1}}\RA{}\nwendquote} entries in parallel.
We supply a target to easily use our {\Tt{}\LA{}latexmkrc~{\nwtagstyle{}\subpageref{NW4GQOIU-3qE2M6-1}}\RA{}\nwendquote} with latexmk(1).
\nwenddocs{}\nwbegincode{25}\sublabel{NW4GQOIU-4Jzoxc-1}\nwmargintag{{\nwtagstyle{}\subpageref{NW4GQOIU-4Jzoxc-1}}}\moddef{target for latexmkrc~{\nwtagstyle{}\subpageref{NW4GQOIU-4Jzoxc-1}}}\endmoddef\nwstartdeflinemarkup\nwusesondefline{\\{NW4GQOIU-4QwfwR-1}}\nwenddeflinemarkup
latexmkrc:
  [ -e $@ -o "$\{INCLUDE_MAKEFILES\}" = "." ] || \\
  $\{LN\} -s $\{INCLUDE_MAKEFILES\}/latexmkrc $@
\nwused{\\{NW4GQOIU-4QwfwR-1}}\nwendcode{}\nwbegindocs{26}We also add the corresponding line for cleaning.
\nwenddocs{}\nwbegincode{27}\sublabel{NW4GQOIU-1Hidyl-1}\nwmargintag{{\nwtagstyle{}\subpageref{NW4GQOIU-1Hidyl-1}}}\moddef{distclean recipe~{\nwtagstyle{}\subpageref{NW4GQOIU-1Hidyl-1}}}\endmoddef\nwstartdeflinemarkup\nwusesondefline{\\{NW4GQOIU-LrkAl-1}}\nwenddeflinemarkup
[ ! -L latexmkrc ] || $\{RM\} latexmkrc
\nwused{\\{NW4GQOIU-LrkAl-1}}\nwendcode{}\nwbegindocs{28}\nwdocspar

We will now discuss the different files we need latex(1) to generate.
Note that in many cases we want latex(1) to generate more files, \eg {\Tt{}.toc\nwendquote}
files, but we do not have to care about these here.
The reason we can ignore those files is that they do not require any external 
tool, \eg bibtex(1), to be run, these files just requires another run of 
latex(1).

\subsection{Auxillary files}

Many steps in compiling a LaTeX document needs the {\Tt{}.aux\nwendquote} file.
Thus we will first introduce a rule for creating the {\Tt{}.aux\nwendquote} file.
We will create it in the specified output directory.
\nwenddocs{}\nwbegincode{29}\sublabel{NW4GQOIU-SIrPU-1}\nwmargintag{{\nwtagstyle{}\subpageref{NW4GQOIU-SIrPU-1}}}\moddef{auxillary files~{\nwtagstyle{}\subpageref{NW4GQOIU-SIrPU-1}}}\endmoddef\nwstartdeflinemarkup\nwusesondefline{\\{NW4GQOIU-4QwfwR-1}}\nwenddeflinemarkup
$\{TEX_OUTDIR\}/%.aux: %.tex
  \LA{}create output directory and preprocess TeX~{\nwtagstyle{}\subpageref{NW4GQOIU-i3X8w-1}}\RA{}
\nwused{\\{NW4GQOIU-4QwfwR-1}}\nwendcode{}\nwbegindocs{30}Then we simply do
\nwenddocs{}\nwbegincode{31}\sublabel{NW4GQOIU-i3X8w-1}\nwmargintag{{\nwtagstyle{}\subpageref{NW4GQOIU-i3X8w-1}}}\moddef{create output directory and preprocess TeX~{\nwtagstyle{}\subpageref{NW4GQOIU-i3X8w-1}}}\endmoddef\nwstartdeflinemarkup\nwusesondefline{\\{NW4GQOIU-SIrPU-1}\\{NW4GQOIU-fQI0z-1}\\{NW4GQOIU-UTABh-1}\\{NW4GQOIU-UTABh-4}}\nwenddeflinemarkup
$\{MKDIR\} $\{TEX_OUTDIR\}
$\{PREPROCESS.tex\}
\nwused{\\{NW4GQOIU-SIrPU-1}\\{NW4GQOIU-fQI0z-1}\\{NW4GQOIU-UTABh-1}\\{NW4GQOIU-UTABh-4}}\nwendcode{}\nwbegindocs{32}\nwdocspar

\subsection{Bibliographies}

One file that is commonly needed is the one used to create the bibliography, 
the {\Tt{}.bbl\nwendquote} file.
There are two ways to create this file, either using classical bibtex(1) or 
using the {\Tt{}biblatex\nwendquote} package and biber(1).
\nwenddocs{}\nwbegincode{33}\sublabel{NW4GQOIU-3g0BAK-1}\nwmargintag{{\nwtagstyle{}\subpageref{NW4GQOIU-3g0BAK-1}}}\moddef{bibliography files~{\nwtagstyle{}\subpageref{NW4GQOIU-3g0BAK-1}}}\endmoddef\nwstartdeflinemarkup\nwusesondefline{\\{NW4GQOIU-4QwfwR-1}}\nwprevnextdefs{\relax}{NW4GQOIU-3g0BAK-2}\nwenddeflinemarkup
\LA{}bbl target for bibtex~{\nwtagstyle{}\subpageref{NW4GQOIU-44dHyo-1}}\RA{}
\LA{}bbl target for biber~{\nwtagstyle{}\subpageref{NW4GQOIU-fQI0z-1}}\RA{}
\nwalsodefined{\\{NW4GQOIU-3g0BAK-2}\\{NW4GQOIU-3g0BAK-3}}\nwused{\\{NW4GQOIU-4QwfwR-1}}\nwendcode{}\nwbegindocs{34}\nwdocspar

The first approach, using bibtex(1), depends on the {\Tt{}.aux\nwendquote} file.
This means that we can have a target creating the desired {\Tt{}.bbl\nwendquote} file from 
the {\Tt{}.aux\nwendquote} file.
\nwenddocs{}\nwbegincode{35}\sublabel{NW4GQOIU-44dHyo-1}\nwmargintag{{\nwtagstyle{}\subpageref{NW4GQOIU-44dHyo-1}}}\moddef{bbl target for bibtex~{\nwtagstyle{}\subpageref{NW4GQOIU-44dHyo-1}}}\endmoddef\nwstartdeflinemarkup\nwusesondefline{\\{NW4GQOIU-3g0BAK-1}}\nwenddeflinemarkup
$\{TEX_OUTDIR\}/%.bbl: $\{TEX_OUTDIR\}/%.aux
  $\{BIBLIOGRAPHY.aux\}
  $\{MV\} $@ $\{@:.bbl=.blg\} $\{TEX_OUTDIR\}
\nwused{\\{NW4GQOIU-3g0BAK-1}}\nwendcode{}\nwbegindocs{36}\nwdocspar

The second approach uses the {\Tt{}biblatex\nwendquote} package~\cite{biblatex} and 
biber(1)~\cite{biber}.
They do not rely on the {\Tt{}.aux\nwendquote} file, instead {\Tt{}biblatex\nwendquote} creates a {\Tt{}.bcf\nwendquote} 
file.
Thus its target is exactly the same as that of the {\Tt{}.aux\nwendquote} file.
\nwenddocs{}\nwbegincode{37}\sublabel{NW4GQOIU-fQI0z-1}\nwmargintag{{\nwtagstyle{}\subpageref{NW4GQOIU-fQI0z-1}}}\moddef{bbl target for biber~{\nwtagstyle{}\subpageref{NW4GQOIU-fQI0z-1}}}\endmoddef\nwstartdeflinemarkup\nwusesondefline{\\{NW4GQOIU-3g0BAK-1}}\nwprevnextdefs{\relax}{NW4GQOIU-fQI0z-2}\nwenddeflinemarkup
$\{TEX_OUTDIR\}/%.bcf: %.tex
  \LA{}create output directory and preprocess TeX~{\nwtagstyle{}\subpageref{NW4GQOIU-i3X8w-1}}\RA{}
\nwalsodefined{\\{NW4GQOIU-fQI0z-2}}\nwused{\\{NW4GQOIU-3g0BAK-1}}\nwendcode{}\nwbegindocs{38}This {\Tt{}.bcf\nwendquote} file is in turn used by biber(1) to create the {\Tt{}.bbl\nwendquote} file.
To compile the {\Tt{}.bbl\nwendquote} with biber(1) and put the output files in the desired
directory, we do the following.
\nwenddocs{}\nwbegincode{39}\sublabel{NW4GQOIU-fQI0z-2}\nwmargintag{{\nwtagstyle{}\subpageref{NW4GQOIU-fQI0z-2}}}\moddef{bbl target for biber~{\nwtagstyle{}\subpageref{NW4GQOIU-fQI0z-1}}}\plusendmoddef\nwstartdeflinemarkup\nwusesondefline{\\{NW4GQOIU-3g0BAK-1}}\nwprevnextdefs{NW4GQOIU-fQI0z-1}{\relax}\nwenddeflinemarkup
$\{TEX_OUTDIR\}/%.bbl: $\{TEX_OUTDIR\}/%.bcf
  $\{BIBLIOGRAPHY.bcf\}
\nwused{\\{NW4GQOIU-3g0BAK-1}}\nwendcode{}\nwbegindocs{40}\nwdocspar

As mentioned in \cref{tex:Intro}, we can automatically add the {\Tt{}.bbl\nwendquote} file as
a prerequisite if the variable {\Tt{}TEX{\_}BBL\nwendquote} is set.
\nwenddocs{}\nwbegincode{41}\sublabel{NW4GQOIU-3g0BAK-2}\nwmargintag{{\nwtagstyle{}\subpageref{NW4GQOIU-3g0BAK-2}}}\moddef{bibliography files~{\nwtagstyle{}\subpageref{NW4GQOIU-3g0BAK-1}}}\plusendmoddef\nwstartdeflinemarkup\nwusesondefline{\\{NW4GQOIU-4QwfwR-1}}\nwprevnextdefs{NW4GQOIU-3g0BAK-1}{NW4GQOIU-3g0BAK-3}\nwenddeflinemarkup
ifneq ($\{TEX_BBL\},)
%.pdf $\{TEX_OUTDIR\}/%.pdf: $\{TEX_OUTDIR\}/%.bbl
endif
\nwused{\\{NW4GQOIU-4QwfwR-1}}\nwendcode{}\nwbegindocs{42}\nwdocspar

\subsection{Indices}

There are several time we need to work with indices, \eg when working with 
standard indices but also glossaries.
Here we provide some suffix rules to make it easier to build such indices.

Before we start, however, we will note that many of these rules are not needed 
if the {\Tt{}imakeidx\nwendquote} package~\cite{imakeidx} is used.
We do recommend to use this package.
Furthermore, we provide rules for the {\Tt{}nomencl\nwendquote} package, however, we 
recommend to use the {\Tt{}glossaries\nwendquote} package~\cite{glossaries} instead.
The {\Tt{}glossaries\nwendquote} package also has native support for xindy(1).
Although the {\Tt{}glossaries\nwendquote} package supports abbreviations and acronyms, we 
recommend the {\Tt{}acro\nwendquote} package~\cite{acro} for this instead.

The standard LaTeX index uses an {\Tt{}.idx\nwendquote} file, which is generated similarly as
the {\Tt{}.aux\nwendquote} file.
Thus we can use the same type of target.
\nwenddocs{}\nwbegincode{43}\sublabel{NW4GQOIU-UTABh-1}\nwmargintag{{\nwtagstyle{}\subpageref{NW4GQOIU-UTABh-1}}}\moddef{indices files~{\nwtagstyle{}\subpageref{NW4GQOIU-UTABh-1}}}\endmoddef\nwstartdeflinemarkup\nwusesondefline{\\{NW4GQOIU-4QwfwR-1}}\nwprevnextdefs{\relax}{NW4GQOIU-UTABh-2}\nwenddeflinemarkup
$\{TEX_OUTDIR\}/%.idx: %.tex
  \LA{}create output directory and preprocess TeX~{\nwtagstyle{}\subpageref{NW4GQOIU-i3X8w-1}}\RA{}
\nwalsodefined{\\{NW4GQOIU-UTABh-2}\\{NW4GQOIU-UTABh-3}\\{NW4GQOIU-UTABh-4}}\nwused{\\{NW4GQOIU-4QwfwR-1}}\nwendcode{}\nwbegindocs{44}The actual index, which resides in a {\Tt{}.ind\nwendquote} file, can then be generated as 
follows.
\nwenddocs{}\nwbegincode{45}\sublabel{NW4GQOIU-UTABh-2}\nwmargintag{{\nwtagstyle{}\subpageref{NW4GQOIU-UTABh-2}}}\moddef{indices files~{\nwtagstyle{}\subpageref{NW4GQOIU-UTABh-1}}}\plusendmoddef\nwstartdeflinemarkup\nwusesondefline{\\{NW4GQOIU-4QwfwR-1}}\nwprevnextdefs{NW4GQOIU-UTABh-1}{NW4GQOIU-UTABh-3}\nwenddeflinemarkup
$\{TEX_OUTDIR\}/%.ind: $\{TEX_OUTDIR\}/%.idx
  $\{COMPILE.idx\}
\nwused{\\{NW4GQOIU-4QwfwR-1}}\nwendcode{}\nwbegindocs{46}As mentioned in \cref{tex:Intro}, we can automatically add the {\Tt{}.ind\nwendquote} file 
as
a prerequisite if the variable {\Tt{}TEX{\_}IND\nwendquote} is set.
\nwenddocs{}\nwbegincode{47}\sublabel{NW4GQOIU-UTABh-3}\nwmargintag{{\nwtagstyle{}\subpageref{NW4GQOIU-UTABh-3}}}\moddef{indices files~{\nwtagstyle{}\subpageref{NW4GQOIU-UTABh-1}}}\plusendmoddef\nwstartdeflinemarkup\nwusesondefline{\\{NW4GQOIU-4QwfwR-1}}\nwprevnextdefs{NW4GQOIU-UTABh-2}{NW4GQOIU-UTABh-4}\nwenddeflinemarkup
ifneq ($\{TEX_IND\},)
%.pdf $\{TEX_OUTDIR\}/%.pdf: $\{TEX_OUTDIR\}/%.ind
endif
\nwused{\\{NW4GQOIU-4QwfwR-1}}\nwendcode{}\nwbegindocs{48}\nwdocspar

For backwards compatibility, we provide the following code for the 
{\Tt{}nomenclature\nwendquote} package.
\nwenddocs{}\nwbegincode{49}\sublabel{NW4GQOIU-UTABh-4}\nwmargintag{{\nwtagstyle{}\subpageref{NW4GQOIU-UTABh-4}}}\moddef{indices files~{\nwtagstyle{}\subpageref{NW4GQOIU-UTABh-1}}}\plusendmoddef\nwstartdeflinemarkup\nwusesondefline{\\{NW4GQOIU-4QwfwR-1}}\nwprevnextdefs{NW4GQOIU-UTABh-3}{\relax}\nwenddeflinemarkup
$\{TEX_OUTDIR\}/%.nlo: %.tex
  \LA{}create output directory and preprocess TeX~{\nwtagstyle{}\subpageref{NW4GQOIU-i3X8w-1}}\RA{}

$\{TEX_OUTDIR\}/%.nls: $\{TEX_OUTDIR\}/%.nlo
  $\{COMPILE.nlo\}
\nwused{\\{NW4GQOIU-4QwfwR-1}}\nwendcode{}\nwbegindocs{50}And now we add the corresponding code for latexmk(1).
The code is fetched from the latexmk example-files on \ac{CTAN}\footnote{%
  URL: \url{http://mirrors.ctan.org/support/latexmk/example_rcfiles/nomenclature_latexmkrc}
}.
\nwenddocs{}\nwbegincode{51}\sublabel{NW4GQOIU-3qE2M6-1}\nwmargintag{{\nwtagstyle{}\subpageref{NW4GQOIU-3qE2M6-1}}}\moddef{latexmkrc~{\nwtagstyle{}\subpageref{NW4GQOIU-3qE2M6-1}}}\endmoddef\nwstartdeflinemarkup\nwprevnextdefs{\relax}{NW4GQOIU-3qE2M6-2}\nwenddeflinemarkup
add_cus_dep( 'nlo', 'nls', 0, 'makenlo2nls' );
sub makenlo2nls \{
  system( "makeindex -s nomencl.ist -o \\"$_[0].nls\\" \\"$_[0].nlo\\"" );
\}
\nwalsodefined{\\{NW4GQOIU-3qE2M6-2}}\nwnotused{latexmkrc}\nwendcode{}\nwbegindocs{52}\nwdocspar

\subsection{PythonTeX}

Occasionally we use PythonTeX.
We also provide a target for the required files.
However, this construction requires that we load PythonTeX as follows.
\begin{minted}{LaTeX}
\usepackage{pythontex}
\setpythontexoutputdir{.}
\setpythontexworkingdir{..}
\end{minted}

The code is as follows.
\nwenddocs{}\nwbegincode{53}\sublabel{NW4GQOIU-287s1n-1}\nwmargintag{{\nwtagstyle{}\subpageref{NW4GQOIU-287s1n-1}}}\moddef{PythonTeX files~{\nwtagstyle{}\subpageref{NW4GQOIU-287s1n-1}}}\endmoddef\nwstartdeflinemarkup\nwusesondefline{\\{NW4GQOIU-4QwfwR-1}}\nwenddeflinemarkup
$\{TEX_OUTDIR\}/%.pytxcode: $\{TEX_OUTDIR\}/%.aux
  cd $(dir $@) && $\{PYTHONTEX\} $\{PYTHONTEXFLAGS\} $(basename $(notdir $@))
%.pytxmcr $\{TEX_OUTDIR\}/%.pytxmcr:: $\{TEX_OUTDIR\}/%.pytxcode
  cd $\{TEX_OUTDIR\} && $\{PYTHONTEX\} $\{PYTHONTEXFLAGS\} $(basename $(notdir $@))
\nwused{\\{NW4GQOIU-4QwfwR-1}}\nwendcode{}\nwbegindocs{54}\nwdocspar

As mentioned in \cref{tex:Intro}, we can automatically add the {\Tt{}.pytxmcr\nwendquote} 
file as a prerequisite if the variable {\Tt{}TEX{\_}PYTHONTEX\nwendquote} is set.
\nwenddocs{}\nwbegincode{55}\sublabel{NW4GQOIU-3g0BAK-3}\nwmargintag{{\nwtagstyle{}\subpageref{NW4GQOIU-3g0BAK-3}}}\moddef{bibliography files~{\nwtagstyle{}\subpageref{NW4GQOIU-3g0BAK-1}}}\plusendmoddef\nwstartdeflinemarkup\nwusesondefline{\\{NW4GQOIU-4QwfwR-1}}\nwprevnextdefs{NW4GQOIU-3g0BAK-2}{\relax}\nwenddeflinemarkup
ifneq ($\{TEX_PYTHONTEX\},)
$\{TEX_OUTDIR\}/%.pdf: $\{TEX_OUTDIR\}/%.pytxmcr
endif
\nwused{\\{NW4GQOIU-4QwfwR-1}}\nwendcode{}\nwbegindocs{56}\nwdocspar

If we use latexmk(1), then we must also add instructions for this in 
{\Tt{}\LA{}latexmkrc~{\nwtagstyle{}\subpageref{NW4GQOIU-3qE2M6-1}}\RA{}\nwendquote}.
The following code is fetched from the latexmk example-files on 
\ac{CTAN}\footnote{%
  URL: \url{http://mirrors.ctan.org/support/latexmk/example_rcfiles/pythontex-latexmkrc}
}.
\nwenddocs{}\nwbegincode{57}\sublabel{NW4GQOIU-3qE2M6-2}\nwmargintag{{\nwtagstyle{}\subpageref{NW4GQOIU-3qE2M6-2}}}\moddef{latexmkrc~{\nwtagstyle{}\subpageref{NW4GQOIU-3qE2M6-1}}}\plusendmoddef\nwstartdeflinemarkup\nwprevnextdefs{NW4GQOIU-3qE2M6-1}{\relax}\nwenddeflinemarkup
#  This version has a fudge on the latex and pdflatex commands that
#  allows the pythontex custom dependency to work even when $out_dir
#  is used to set the output directory.  Without the fudge (done by
#  trickery symbolic links) the custom dependency for using pythontex
#  will not be detected.

add_cus_dep('pytxcode', 'pytxmcr', 0, 'pythontex');
sub pythontex \{
    # This subroutine is a fudge, because it from latexmk's point of
    # view, it makes the main .tex file depend on the .pytxcode file.
    # But it doesn't actually make the .tex file, but is used for its
    # side effects in creating other files.  The dependence is a way
    # of triggering the rule to be run whenever the .pytxcode file
    # changes, and to do this before running latex/pdflatex again.
    return system("pythontex3 --verbose \\"$_[0]\\"");
\}

$pdflatex = 'internal mylatex %R %Z pdflatex %O %S';
$latex = 'internal mylatex %R %Z latex %O %S';
sub mylatex \{
   my $root = shift;
   my $dir_string = shift;
   my $code = "$root.pytxcode";
   my $result = "pythontex-files-$root";
   if ($dir_string) \{
      warn "mylatex: Making symlinks to fool cus_dep creation\\n";
      unlink $code;
      if (-l $result) \{
          unlink $result;
      \}
      elsif (-d $result) \{
         unlink glob "$result/*";
         rmdir $result;
      \}
      symlink $dir_string.$code, $code;
      if ( ! -e $dir_string.$result ) \{ mkdir $dir_string.$result; \}
      symlink $dir_string.$result, $result;
   \}
   else \{
      foreach ($code, $result) \{ if (-l) \{ unlink; \} \}
   \}
   return system @_;
\}
\nwendcode{}\nwbegindocs{58}\nwdocspar

\subsection{Document files}

Now that we have all prerequisite files, we can actually compile the document.
For simplicity we add file in both the current working directory and the 
{\Tt{}TEX{\_}OUTDIR\nwendquote} as targets.
The reason for this is that it makes the makefile easier to write, usually we 
prefer writing just the {\Tt{}.pdf\nwendquote} file --- and not the path to the {\Tt{}.pdf\nwendquote} in 
the {\Tt{}TEX{\_}OUTDIR\nwendquote} directory.
And for the same reason, we create a symbolic link between them after 
compilation --- this allows make(1) to track modification times correctly.
(We need a symbolic link, rather than a hard link, because upon recompilation, 
the file in {\Tt{}TEX{\_}OUTDIR\nwendquote} is unlinked, whereas the other is not.)
\nwenddocs{}\nwbegincode{59}\sublabel{NW4GQOIU-2cBbLf-1}\nwmargintag{{\nwtagstyle{}\subpageref{NW4GQOIU-2cBbLf-1}}}\moddef{document files~{\nwtagstyle{}\subpageref{NW4GQOIU-2cBbLf-1}}}\endmoddef\nwstartdeflinemarkup\nwusesondefline{\\{NW4GQOIU-4QwfwR-1}}\nwenddeflinemarkup
%.pdf $\{TEX_OUTDIR\}/%.pdf: %.tex
  $\{COMPILE.tex\}
  -$\{LN\} $\{TEX_OUTDIR\}/$@ $@

%.dvi $\{TEX_OUTDIR\}/%.dvi: %.tex
  \LA{}compile DVI~{\nwtagstyle{}\subpageref{NW4GQOIU-whl11-1}}\RA{}
  -$\{LN\} $\{TEX_OUTDIR\}/$@ $@
\nwused{\\{NW4GQOIU-4QwfwR-1}}\nwendcode{}\nwbegindocs{60}\nwdocspar

The compilation step for DVI is the usual.
We compile once, then we recompile as long as the log file tells us.
\nwenddocs{}\nwbegincode{61}\sublabel{NW4GQOIU-whl11-1}\nwmargintag{{\nwtagstyle{}\subpageref{NW4GQOIU-whl11-1}}}\moddef{compile DVI~{\nwtagstyle{}\subpageref{NW4GQOIU-whl11-1}}}\endmoddef\nwstartdeflinemarkup\nwusesondefline{\\{NW4GQOIU-2cBbLf-1}\\{NW4GQOIU-1s6Szf-1}}\nwenddeflinemarkup
$\{LATEX\} -output-directory=$\{TEX_OUTDIR\} $\{LATEXFLAGS\} $<
while ( grep "Rerun to get cross" $\{TEX_OUTDIR\}/$\{<:.tex=.log\} ); do \\
  $\{LATEX\} -output-directory=$\{TEX_OUTDIR\} $\{LATEXFLAGS\} $<; \\
done
\nwused{\\{NW4GQOIU-2cBbLf-1}\\{NW4GQOIU-1s6Szf-1}}\nwendcode{}\nwbegindocs{62}\nwdocspar


\section{Targets for merging bibliographies}

To merge bibliographies we will use {\Tt{}bibtool\nwendquote} and make's archive rules.
In particular, we extend the general rules we defined in {\Tt{}portability.mk\nwendquote}.
\nwenddocs{}\nwbegincode{63}\sublabel{NW4GQOIU-29jjeg-9}\nwmargintag{{\nwtagstyle{}\subpageref{NW4GQOIU-29jjeg-9}}}\moddef{variables~{\nwtagstyle{}\subpageref{NW4GQOIU-29jjeg-1}}}\plusendmoddef\nwstartdeflinemarkup\nwusesondefline{\\{NW4GQOIU-2kaaDB-1}}\nwprevnextdefs{NW4GQOIU-29jjeg-8}{\relax}\nwenddeflinemarkup
BIBTOOL?=     bibtool
BIBTOOLFLAGS?=--preserve.key.case=on --print.deleted.entries=off -s -d -r biblatex
ARCHIVE.bib?= $\{CAT\} $(if $(wildcard $@),$@) $% | \\
  $\{BIBTOOL\} $\{BIBTOOLFLAGS\} -o $@
\nwused{\\{NW4GQOIU-2kaaDB-1}}\nwendcode{}\nwbegindocs{64}\nwdocspar


\section{Targets for class and package files}

There are two parts concerning class and package files.
\nwenddocs{}\nwbegincode{65}\sublabel{NW4GQOIU-tejPg-1}\nwmargintag{{\nwtagstyle{}\subpageref{NW4GQOIU-tejPg-1}}}\moddef{targets for class and package files~{\nwtagstyle{}\subpageref{NW4GQOIU-tejPg-1}}}\endmoddef\nwstartdeflinemarkup\nwusesondefline{\\{NW4GQOIU-2kaaDB-1}}\nwenddeflinemarkup
\LA{}compile sty and cls files~{\nwtagstyle{}\subpageref{NW4GQOIU-cfvLA-1}}\RA{}
\LA{}compile class and package documentation~{\nwtagstyle{}\subpageref{NW4GQOIU-1s6Szf-1}}\RA{}
\nwused{\\{NW4GQOIU-2kaaDB-1}}\nwendcode{}\nwbegindocs{66}These are very similar to what we have done above, especially the 
documentation.

Compiling a class or package from DocTeX source is easier than compiling 
a document.
We can normally create the {\Tt{}.sty\nwendquote} and {\Tt{}.cls\nwendquote} files by running latex(1) on 
the {\Tt{}.ins\nwendquote} file.
\nwenddocs{}\nwbegincode{67}\sublabel{NW4GQOIU-cfvLA-1}\nwmargintag{{\nwtagstyle{}\subpageref{NW4GQOIU-cfvLA-1}}}\moddef{compile sty and cls files~{\nwtagstyle{}\subpageref{NW4GQOIU-cfvLA-1}}}\endmoddef\nwstartdeflinemarkup\nwusesondefline{\\{NW4GQOIU-tejPg-1}}\nwenddeflinemarkup
%.cls %.sty: %.ins
  $\{LATEX\} $<
\nwused{\\{NW4GQOIU-tejPg-1}}\nwendcode{}\nwbegindocs{68}\nwdocspar

We can then compile the documentation similarly to how we compile normal 
documents.
\nwenddocs{}\nwbegincode{69}\sublabel{NW4GQOIU-1s6Szf-1}\nwmargintag{{\nwtagstyle{}\subpageref{NW4GQOIU-1s6Szf-1}}}\moddef{compile class and package documentation~{\nwtagstyle{}\subpageref{NW4GQOIU-1s6Szf-1}}}\endmoddef\nwstartdeflinemarkup\nwusesondefline{\\{NW4GQOIU-tejPg-1}}\nwprevnextdefs{\relax}{NW4GQOIU-1s6Szf-2}\nwenddeflinemarkup
%.pdf $\{TEX_OUTDIR\}/%.pdf: %.dtx
  $\{COMPILE.dtx\}
  -$\{LN\} $\{TEX_OUTDIR\}/$@ $@

%.dvi $\{TEX_OUTDIR\}/%.dvi: %.dtx
  \LA{}compile DVI~{\nwtagstyle{}\subpageref{NW4GQOIU-whl11-1}}\RA{}
  -$\{LN\} $\{TEX_OUTDIR\}/$@ $@
\nwalsodefined{\\{NW4GQOIU-1s6Szf-2}}\nwused{\\{NW4GQOIU-tejPg-1}}\nwendcode{}\nwbegindocs{70}However, we must tell make(1) how to make a {\Tt{}.bbl\nwendquote} etc.\ from {\Tt{}.dtx\nwendquote}.
\nwenddocs{}\nwbegincode{71}\sublabel{NW4GQOIU-1s6Szf-2}\nwmargintag{{\nwtagstyle{}\subpageref{NW4GQOIU-1s6Szf-2}}}\moddef{compile class and package documentation~{\nwtagstyle{}\subpageref{NW4GQOIU-1s6Szf-1}}}\plusendmoddef\nwstartdeflinemarkup\nwusesondefline{\\{NW4GQOIU-tejPg-1}}\nwprevnextdefs{NW4GQOIU-1s6Szf-1}{\relax}\nwenddeflinemarkup
$\{TEX_OUTDIR\}/%.aux $\{TEX_OUTDIR\}/%.bcf $\{TEX_OUTDIR\}/%.idx: %.dtx
  $\{PREPROCESS.dtx\}
\nwused{\\{NW4GQOIU-tejPg-1}}\nwendcode{}\nwbegindocs{72}\nwdocspar


\section{External classes and packages}

Occasionally, we are required to use document classes that are not in 
\ac{CTAN}.
Here we provide targets for some such classes and packages.
\nwenddocs{}\nwbegincode{73}\sublabel{NW4GQOIU-2Oad4t-1}\nwmargintag{{\nwtagstyle{}\subpageref{NW4GQOIU-2Oad4t-1}}}\moddef{targets for external classes~{\nwtagstyle{}\subpageref{NW4GQOIU-2Oad4t-1}}}\endmoddef\nwstartdeflinemarkup\nwusesondefline{\\{NW4GQOIU-2kaaDB-1}}\nwenddeflinemarkup
\LA{}a general downloader~{\nwtagstyle{}\subpageref{NW4GQOIU-1Zq3gR-1}}\RA{}
\LA{}Springer LNCS~{\nwtagstyle{}\subpageref{NW4GQOIU-42CEER-1}}\RA{}
\LA{}biblatex LNCS style~{\nwtagstyle{}\subpageref{NW4GQOIU-JX9cs-1}}\RA{}
\LA{}the RFC bibliography~{\nwtagstyle{}\subpageref{NW4GQOIU-1nm7wo-1}}\RA{}
\LA{}PoPETS~{\nwtagstyle{}\subpageref{NW4GQOIU-347V4M-1}}\RA{}
\nwused{\\{NW4GQOIU-2kaaDB-1}}\nwendcode{}\nwbegindocs{74}\nwdocspar

We will now construct a general downloader, then we will use this downloader to
write targets for the external classes we are interested in.
In general, what we want this function to do is to download an archive or 
repository ({\Tt{}TEX{\_}EXT{\_}SRC\nwendquote}), extract the files we are interested in 
({\Tt{}TEX{\_}EXT{\_}FILES\nwendquote}) to the destination directory ({\Tt{}TEX{\_}EXT{\_}DIR\nwendquote}), and 
finally, create symbolic links to those files from the current working 
directory.
\nwenddocs{}\nwbegincode{75}\sublabel{NW4GQOIU-1Zq3gR-1}\nwmargintag{{\nwtagstyle{}\subpageref{NW4GQOIU-1Zq3gR-1}}}\moddef{a general downloader~{\nwtagstyle{}\subpageref{NW4GQOIU-1Zq3gR-1}}}\endmoddef\nwstartdeflinemarkup\nwusesondefline{\\{NW4GQOIU-2Oad4t-1}}\nwenddeflinemarkup
define download_archive
\LA{}targets for symlinks~{\nwtagstyle{}\subpageref{NW4GQOIU-1DUTgs-1}}\RA{}
\LA{}targets for desired files~{\nwtagstyle{}\subpageref{NW4GQOIU-22rOUj-1}}\RA{}
\LA{}targets for archive~{\nwtagstyle{}\subpageref{NW4GQOIU-3iv42z-1}}\RA{}
\LA{}target for cleaning~{\nwtagstyle{}\subpageref{NW4GQOIU-YyAiu-1}}\RA{}
endef
define download_repo
\LA{}targets for symlinks~{\nwtagstyle{}\subpageref{NW4GQOIU-1DUTgs-1}}\RA{}
\LA{}targets for desired files~{\nwtagstyle{}\subpageref{NW4GQOIU-22rOUj-1}}\RA{}
\LA{}targets for repo~{\nwtagstyle{}\subpageref{NW4GQOIU-4eSKbD-1}}\RA{}
\LA{}target for cleaning~{\nwtagstyle{}\subpageref{NW4GQOIU-YyAiu-1}}\RA{}
endef
\nwused{\\{NW4GQOIU-2Oad4t-1}}\nwendcode{}\nwbegindocs{76}\nwdocspar

The first thing we want to do is to generate targets and recipes for how to 
create the symbolic links, assuming that the target files already exists.
First we set up a dependency between the files we are interested in and where 
that file is actually located.
Then we create a recipe which will create a symbolic link between them.
We use the {\Tt{}notdir\nwendquote} function~\cite[Sect.\ 8.3]{GNUMake} to remove any 
directory-part since we want to create the symbolic link in the current working
directory.
\nwenddocs{}\nwbegincode{77}\sublabel{NW4GQOIU-1DUTgs-1}\nwmargintag{{\nwtagstyle{}\subpageref{NW4GQOIU-1DUTgs-1}}}\moddef{targets for symlinks~{\nwtagstyle{}\subpageref{NW4GQOIU-1DUTgs-1}}}\endmoddef\nwstartdeflinemarkup\nwusesondefline{\\{NW4GQOIU-1Zq3gR-1}}\nwprevnextdefs{\relax}{NW4GQOIU-1DUTgs-2}\nwenddeflinemarkup
$(foreach file,$\{TEX_EXT_FILES-$(1)\},\\
  $(eval $(notdir $\{file\}): $\{TEX_EXT_DIR-$(1)\}/$\{file\}))
$(notdir $\{TEX_EXT_FILES-$(1)\}):
  $\{LN\} $$^ $$@
\nwalsodefined{\\{NW4GQOIU-1DUTgs-2}}\nwused{\\{NW4GQOIU-1Zq3gR-1}}\nwendcode{}\nwbegindocs{78}Note that we need the {\Tt{}eval\nwendquote} command above to evaluate the rule for each 
file, otherwise we would get \emph{one line} with many colons --- which is not 
valid syntax for make(1).
To make it easier to add these files as prerequisites to a target, we also 
provide the following phony target.
\nwenddocs{}\nwbegincode{79}\sublabel{NW4GQOIU-1DUTgs-2}\nwmargintag{{\nwtagstyle{}\subpageref{NW4GQOIU-1DUTgs-2}}}\moddef{targets for symlinks~{\nwtagstyle{}\subpageref{NW4GQOIU-1DUTgs-1}}}\plusendmoddef\nwstartdeflinemarkup\nwusesondefline{\\{NW4GQOIU-1Zq3gR-1}}\nwprevnextdefs{NW4GQOIU-1DUTgs-1}{\relax}\nwenddeflinemarkup
.PHONY: $(1)
$(1): $(notdir $\{TEX_EXT_FILES-$(1)\})
\nwused{\\{NW4GQOIU-1Zq3gR-1}}\nwendcode{}\nwbegindocs{80}\nwdocspar

Now we need something to trigger the download of the archive or the repository.
One way to do this is to add a prerequisite for the files from the archive or 
repository.
\nwenddocs{}\nwbegincode{81}\sublabel{NW4GQOIU-22rOUj-1}\nwmargintag{{\nwtagstyle{}\subpageref{NW4GQOIU-22rOUj-1}}}\moddef{targets for desired files~{\nwtagstyle{}\subpageref{NW4GQOIU-22rOUj-1}}}\endmoddef\nwstartdeflinemarkup\nwusesondefline{\\{NW4GQOIU-1Zq3gR-1}}\nwenddeflinemarkup
$(addprefix $\{TEX_EXT_DIR-$(1)\}/,$\{TEX_EXT_FILES-$(1)\}): \\
  $\{TEX_EXT_DIR-$(1)\}/$\{TEX_EXT_SRC-$(1)\}
\nwused{\\{NW4GQOIU-1Zq3gR-1}}\nwendcode{}\nwbegindocs{82}Now we turn to the recipe.
In the case of an archive, we must extract the desired files.
We let the variable {\Tt{}TEX{\_}EXT{\_}EXTRACT\nwendquote} contain the extraction command.
\nwenddocs{}\nwbegincode{83}\sublabel{NW4GQOIU-3iv42z-1}\nwmargintag{{\nwtagstyle{}\subpageref{NW4GQOIU-3iv42z-1}}}\moddef{targets for archive~{\nwtagstyle{}\subpageref{NW4GQOIU-3iv42z-1}}}\endmoddef\nwstartdeflinemarkup\nwusesondefline{\\{NW4GQOIU-1Zq3gR-1}}\nwprevnextdefs{\relax}{NW4GQOIU-3iv42z-2}\nwenddeflinemarkup
$(addprefix $\{TEX_EXT_DIR-$(1)\}/,$\{TEX_EXT_FILES-$(1)\}):
  $\{TEX_EXT_EXTRACT-$(1)\}
\nwalsodefined{\\{NW4GQOIU-3iv42z-2}}\nwused{\\{NW4GQOIU-1Zq3gR-1}}\nwendcode{}\nwbegindocs{84}For a repository we can simply copy the file or create a link.
We prefer the latter.
\nwenddocs{}\nwbegincode{85}\sublabel{NW4GQOIU-4eSKbD-1}\nwmargintag{{\nwtagstyle{}\subpageref{NW4GQOIU-4eSKbD-1}}}\moddef{targets for repo~{\nwtagstyle{}\subpageref{NW4GQOIU-4eSKbD-1}}}\endmoddef\nwstartdeflinemarkup\nwusesondefline{\\{NW4GQOIU-1Zq3gR-1}}\nwprevnextdefs{\relax}{NW4GQOIU-4eSKbD-2}\nwenddeflinemarkup
$(addprefix $\{TEX_EXT_DIR-$(1)\}/,$\{TEX_EXT_FILES-$(1)\}):
  $\{LN\} $\{TEX_EXT_SRC-$(1)\}/$$\{@:$\{TEX_EXT_DIR-$(1)\}/%=%\} $$@
\nwalsodefined{\\{NW4GQOIU-4eSKbD-2}}\nwused{\\{NW4GQOIU-1Zq3gR-1}}\nwendcode{}\nwbegindocs{86}\nwdocspar

The file {\Tt{}TEX{\_}EXT{\_}SRC\nwendquote} can be either an archive or a repository.
We let {\Tt{}TEX{\_}EXT{\_}URL\nwendquote} be the \ac{URL} to fetch it from in both cases.
In the case of an archive we do the following.
\nwenddocs{}\nwbegincode{87}\sublabel{NW4GQOIU-3iv42z-2}\nwmargintag{{\nwtagstyle{}\subpageref{NW4GQOIU-3iv42z-2}}}\moddef{targets for archive~{\nwtagstyle{}\subpageref{NW4GQOIU-3iv42z-1}}}\plusendmoddef\nwstartdeflinemarkup\nwusesondefline{\\{NW4GQOIU-1Zq3gR-1}}\nwprevnextdefs{NW4GQOIU-3iv42z-1}{\relax}\nwenddeflinemarkup
$\{TEX_EXT_DIR-$(1)\}/$\{TEX_EXT_SRC-$(1)\}:
  $\{MKDIR\} $\{TEX_EXT_DIR-$(1)\}
  $\{CURL\} -o $$@ $\{TEX_EXT_URL-$(1)\}
\nwused{\\{NW4GQOIU-1Zq3gR-1}}\nwendcode{}\nwbegindocs{88}In the case of a repository, we simply clone it.
\nwenddocs{}\nwbegincode{89}\sublabel{NW4GQOIU-4eSKbD-2}\nwmargintag{{\nwtagstyle{}\subpageref{NW4GQOIU-4eSKbD-2}}}\moddef{targets for repo~{\nwtagstyle{}\subpageref{NW4GQOIU-4eSKbD-1}}}\plusendmoddef\nwstartdeflinemarkup\nwusesondefline{\\{NW4GQOIU-1Zq3gR-1}}\nwprevnextdefs{NW4GQOIU-4eSKbD-1}{\relax}\nwenddeflinemarkup
$\{TEX_EXT_DIR-$(1)\}/$\{TEX_EXT_SRC-$(1)\}:
  git clone $\{TEX_EXT_URL-$(1)\} $$@
\nwused{\\{NW4GQOIU-1Zq3gR-1}}\nwendcode{}\nwbegindocs{90}We note that the directory of the repo should be an order-only 
prerequisite~\cite[see][Sect.\ 4.3]{GNUMake} for the files inside.
Unfortunately this is not the case at the moment.

Finally, we must also do some cleaning.
\nwenddocs{}\nwbegincode{91}\sublabel{NW4GQOIU-YyAiu-1}\nwmargintag{{\nwtagstyle{}\subpageref{NW4GQOIU-YyAiu-1}}}\moddef{target for cleaning~{\nwtagstyle{}\subpageref{NW4GQOIU-YyAiu-1}}}\endmoddef\nwstartdeflinemarkup\nwusesondefline{\\{NW4GQOIU-1Zq3gR-1}}\nwenddeflinemarkup
.PHONY: distclean clean-$(1)
distclean: clean-$(1)
clean-$(1):
  $\{RM\} $\{TEX_EXT_FILES-$(1)\}
  [ "$\{TEX_EXT_DIR-$(1)\}" = "." ] && $\{RM\} $\{TEX_EXT_SRC-$(1)\} \\
    || $\{RM\} -R $\{TEX_EXT_DIR-$(1)\}
\nwused{\\{NW4GQOIU-1Zq3gR-1}}\nwendcode{}\nwbegindocs{92}\nwdocspar

\subsection{Springer LNCS}

Springer's \ac{LNCS} series is used for the proceedings of many 
conferences.
The style files are available on Springer's web site, but unfortunately not 
under any permissive license\footnote{%
  It would be most desirable that they were made available in \ac{CTAN} under 
  an open license.
}.
So, we must, each and every one of us, connect to Springer's server and 
download our own copy.
This is what we automate here.

We use the downloader described above.
\nwenddocs{}\nwbegincode{93}\sublabel{NW4GQOIU-42CEER-1}\nwmargintag{{\nwtagstyle{}\subpageref{NW4GQOIU-42CEER-1}}}\moddef{Springer LNCS~{\nwtagstyle{}\subpageref{NW4GQOIU-42CEER-1}}}\endmoddef\nwstartdeflinemarkup\nwusesondefline{\\{NW4GQOIU-2Oad4t-1}}\nwprevnextdefs{\relax}{NW4GQOIU-42CEER-2}\nwenddeflinemarkup
TEX_EXT_FILES-lncs?=  llncs.cls sprmindx.sty splncs03.bst aliascnt.sty remreset.sty
TEX_EXT_DIR-lncs?=    lncs
TEX_EXT_SRC-lncs?=    llncs2e.zip
TEX_EXT_URL-lncs?=    https://resource-cms.springernature.com/springer-cms/rest/v1/content/19238648/data/v1
TEX_EXT_EXTRACT-lncs?=$\{UNZIP\} -u $$< -d $\{TEX_EXT_DIR-lncs\}

$(eval $(call download_archive,lncs))
\nwalsodefined{\\{NW4GQOIU-42CEER-2}}\nwused{\\{NW4GQOIU-2Oad4t-1}}\nwendcode{}\nwbegindocs{94}\nwdocspar

We also want to add backwards compatibility for when we used {\Tt{}llncs\nwendquote} instead 
of just {\Tt{}lncs\nwendquote}.
\nwenddocs{}\nwbegincode{95}\sublabel{NW4GQOIU-42CEER-2}\nwmargintag{{\nwtagstyle{}\subpageref{NW4GQOIU-42CEER-2}}}\moddef{Springer LNCS~{\nwtagstyle{}\subpageref{NW4GQOIU-42CEER-1}}}\plusendmoddef\nwstartdeflinemarkup\nwusesondefline{\\{NW4GQOIU-2Oad4t-1}}\nwprevnextdefs{NW4GQOIU-42CEER-1}{\relax}\nwenddeflinemarkup
.PHONY: llncs
llncs: lncs
\nwused{\\{NW4GQOIU-2Oad4t-1}}\nwendcode{}\nwbegindocs{96}\nwdocspar

\subsection{LNCS style for {\Tt{}biblatex\nwendquote}}

There is also \iac{LNCS} style for the {\Tt{}biblatex\nwendquote} package available on 
GitHub.
Since it is available on GitHub, we recommend adding it as a Git submodule.
I.e.\ run the following command.
\begin{lstlisting}
git submodule add https://github.com/neapel/biblatex-lncs.git
\end{lstlisting}
This will add a directory {\Tt{}biblatex-lncs\nwendquote} to the current directory.

If we do not add it as a submodule we can use the downloader above.
\nwenddocs{}\nwbegincode{97}\sublabel{NW4GQOIU-JX9cs-1}\nwmargintag{{\nwtagstyle{}\subpageref{NW4GQOIU-JX9cs-1}}}\moddef{biblatex LNCS style~{\nwtagstyle{}\subpageref{NW4GQOIU-JX9cs-1}}}\endmoddef\nwstartdeflinemarkup\nwusesondefline{\\{NW4GQOIU-2Oad4t-1}}\nwenddeflinemarkup
TEX_EXT_FILES-biblatex-lncs?= lncs.bbx lncs.cbx lncs.dbx
TEX_EXT_DIR-biblatex-lncs?=   lncs
TEX_EXT_SRC-biblatex-lncs?=   biblatex-lncs
TEX_EXT_URL-biblatex-lncs?=   https://github.com/NorwegianRockCat/biblatex-lncs.git

$(eval $(call download_repo,biblatex-lncs))
\nwused{\\{NW4GQOIU-2Oad4t-1}}\nwendcode{}\nwbegindocs{98}\nwdocspar

\subsection{The RFC bibliography}

Occasionally we want to cite \ac{IETF} \acp{RFC}.
Fortunately, Roland Bless of Karlsruher Institute of Technology provides an 
up-to-date bibliography file for all \acp{RFC}, so we will use that one.
This is a single file, so we do not need to use the downloader.
\nwenddocs{}\nwbegincode{99}\sublabel{NW4GQOIU-1nm7wo-1}\nwmargintag{{\nwtagstyle{}\subpageref{NW4GQOIU-1nm7wo-1}}}\moddef{the RFC bibliography~{\nwtagstyle{}\subpageref{NW4GQOIU-1nm7wo-1}}}\endmoddef\nwstartdeflinemarkup\nwusesondefline{\\{NW4GQOIU-2Oad4t-1}}\nwprevnextdefs{\relax}{NW4GQOIU-1nm7wo-2}\nwenddeflinemarkup
rfc.bib:
  \LA{}download rfc.bib~{\nwtagstyle{}\subpageref{NW4GQOIU-472ZSW-1}}\RA{}
  \LA{}change misc to techreport~{\nwtagstyle{}\subpageref{NW4GQOIU-CO7Lj-1}}\RA{}

$\{TEXMF\}/tex/latex/rfc.bib:
  mkdir -p $\{TEXMF\}/tex/latex/
  \LA{}download rfc.bib~{\nwtagstyle{}\subpageref{NW4GQOIU-472ZSW-1}}\RA{}
  \LA{}change misc to techreport~{\nwtagstyle{}\subpageref{NW4GQOIU-CO7Lj-1}}\RA{}
\nwalsodefined{\\{NW4GQOIU-1nm7wo-2}\\{NW4GQOIU-1nm7wo-3}}\nwused{\\{NW4GQOIU-2Oad4t-1}}\nwendcode{}\nwbegindocs{100}\nwdocspar

We will use curl(1) to download a compressed version from Bless' site.
We let curl(1) output the contents to standard out and pipe it to the 
uncompress(1) utility and, finally, redirect the result to the target file.
\nwenddocs{}\nwbegincode{101}\sublabel{NW4GQOIU-472ZSW-1}\nwmargintag{{\nwtagstyle{}\subpageref{NW4GQOIU-472ZSW-1}}}\moddef{download rfc.bib~{\nwtagstyle{}\subpageref{NW4GQOIU-472ZSW-1}}}\endmoddef\nwstartdeflinemarkup\nwusesondefline{\\{NW4GQOIU-1nm7wo-1}}\nwenddeflinemarkup
$\{CURL\} -o - http://tm.uka.de/~bless/rfc.bib.gz 2>/dev/null \\
  | $\{UNCOMPRESS.gz\} - > $@ ; \\
\nwused{\\{NW4GQOIU-1nm7wo-1}}\nwendcode{}\nwbegindocs{102}\nwdocspar

According to \ac{IETF}~\cite[Sect.\ 5.2]{IETFCitingRFCs} the \acp{RFC} should 
be cited as the {\Tt{}techreport\nwendquote} BibTeX type.
\nwenddocs{}\nwbegincode{103}\sublabel{NW4GQOIU-CO7Lj-1}\nwmargintag{{\nwtagstyle{}\subpageref{NW4GQOIU-CO7Lj-1}}}\moddef{change misc to techreport~{\nwtagstyle{}\subpageref{NW4GQOIU-CO7Lj-1}}}\endmoddef\nwstartdeflinemarkup\nwusesondefline{\\{NW4GQOIU-1nm7wo-1}}\nwenddeflinemarkup
$\{SED\} -i "s/@misc/@techreport/" $@
\nwused{\\{NW4GQOIU-1nm7wo-1}}\nwendcode{}\nwbegindocs{104}\nwdocspar

We also provide a phony target for these two files.
\nwenddocs{}\nwbegincode{105}\sublabel{NW4GQOIU-1nm7wo-2}\nwmargintag{{\nwtagstyle{}\subpageref{NW4GQOIU-1nm7wo-2}}}\moddef{the RFC bibliography~{\nwtagstyle{}\subpageref{NW4GQOIU-1nm7wo-1}}}\plusendmoddef\nwstartdeflinemarkup\nwusesondefline{\\{NW4GQOIU-2Oad4t-1}}\nwprevnextdefs{NW4GQOIU-1nm7wo-1}{NW4GQOIU-1nm7wo-3}\nwenddeflinemarkup
.PHONY: rfc
rfc: rfc.bib $\{TEXMF\}/tex/latex/rfc.bib
\nwused{\\{NW4GQOIU-2Oad4t-1}}\nwendcode{}\nwbegindocs{106}\nwdocspar

Finally, we provide a phony cleaning for cleaning.
The target is named {\Tt{}clean-rfc\nwendquote} and is added as a prerequisite for 
{\Tt{}distclean\nwendquote} --- this way its recipe will not interfere with any cleaning 
recipe written by the user.
\nwenddocs{}\nwbegincode{107}\sublabel{NW4GQOIU-1nm7wo-3}\nwmargintag{{\nwtagstyle{}\subpageref{NW4GQOIU-1nm7wo-3}}}\moddef{the RFC bibliography~{\nwtagstyle{}\subpageref{NW4GQOIU-1nm7wo-1}}}\plusendmoddef\nwstartdeflinemarkup\nwusesondefline{\\{NW4GQOIU-2Oad4t-1}}\nwprevnextdefs{NW4GQOIU-1nm7wo-2}{\relax}\nwenddeflinemarkup
.PHONY: distclean clean-rfc
distclean: clean-rfc
clean-rfc:
  $\{RM\} rfc.bib
\nwused{\\{NW4GQOIU-2Oad4t-1}}\nwendcode{}\nwbegindocs{108}\nwdocspar

\subsection{Proceedings of the Privacy Enhancing Technologies Symposium}

We would also like to be able to use the PoPETS format.
\nwenddocs{}\nwbegincode{109}\sublabel{NW4GQOIU-347V4M-1}\nwmargintag{{\nwtagstyle{}\subpageref{NW4GQOIU-347V4M-1}}}\moddef{PoPETS~{\nwtagstyle{}\subpageref{NW4GQOIU-347V4M-1}}}\endmoddef\nwstartdeflinemarkup\nwusesondefline{\\{NW4GQOIU-2Oad4t-1}}\nwenddeflinemarkup
TEX_EXT_FILES-popets?=by-nc-nd.pdf sciendo-logo.pdf dgruyter_NEW.sty
TEX_EXT_URL-popets?=https://petsymposium.org/files/popets.zip
TEX_EXT_DIR-popets?=popets
TEX_EXT_SRC-popets?=popets.zip
TEX_EXT_EXTRACT-popets?=$\{UNZIP\} -p $$< popets/$$(notdir $$@) > $$@

$(eval $(call download_archive,popets))
\nwused{\\{NW4GQOIU-2Oad4t-1}}\nwendcode{}

\nwixlogsorted{c}{{a general downloader}{NW4GQOIU-1Zq3gR-1}{\nwixu{NW4GQOIU-2Oad4t-1}\nwixd{NW4GQOIU-1Zq3gR-1}}}%
\nwixlogsorted{c}{{auxillary files}{NW4GQOIU-SIrPU-1}{\nwixu{NW4GQOIU-4QwfwR-1}\nwixd{NW4GQOIU-SIrPU-1}}}%
\nwixlogsorted{c}{{bbl target for biber}{NW4GQOIU-fQI0z-1}{\nwixu{NW4GQOIU-3g0BAK-1}\nwixd{NW4GQOIU-fQI0z-1}\nwixd{NW4GQOIU-fQI0z-2}}}%
\nwixlogsorted{c}{{bbl target for bibtex}{NW4GQOIU-44dHyo-1}{\nwixu{NW4GQOIU-3g0BAK-1}\nwixd{NW4GQOIU-44dHyo-1}}}%
\nwixlogsorted{c}{{biblatex LNCS style}{NW4GQOIU-JX9cs-1}{\nwixu{NW4GQOIU-2Oad4t-1}\nwixd{NW4GQOIU-JX9cs-1}}}%
\nwixlogsorted{c}{{bibliography files}{NW4GQOIU-3g0BAK-1}{\nwixu{NW4GQOIU-4QwfwR-1}\nwixd{NW4GQOIU-3g0BAK-1}\nwixd{NW4GQOIU-3g0BAK-2}\nwixd{NW4GQOIU-3g0BAK-3}}}%
\nwixlogsorted{c}{{change misc to techreport}{NW4GQOIU-CO7Lj-1}{\nwixu{NW4GQOIU-1nm7wo-1}\nwixd{NW4GQOIU-CO7Lj-1}}}%
\nwixlogsorted{c}{{clean recipe}{NW4GQOIU-mtljB-1}{\nwixu{NW4GQOIU-LrkAl-1}\nwixd{NW4GQOIU-mtljB-1}}}%
\nwixlogsorted{c}{{compile class and package documentation}{NW4GQOIU-1s6Szf-1}{\nwixu{NW4GQOIU-tejPg-1}\nwixd{NW4GQOIU-1s6Szf-1}\nwixd{NW4GQOIU-1s6Szf-2}}}%
\nwixlogsorted{c}{{compile DVI}{NW4GQOIU-whl11-1}{\nwixu{NW4GQOIU-2cBbLf-1}\nwixd{NW4GQOIU-whl11-1}\nwixu{NW4GQOIU-1s6Szf-1}}}%
\nwixlogsorted{c}{{compile sty and cls files}{NW4GQOIU-cfvLA-1}{\nwixu{NW4GQOIU-tejPg-1}\nwixd{NW4GQOIU-cfvLA-1}}}%
\nwixlogsorted{c}{{create output directory and preprocess TeX}{NW4GQOIU-i3X8w-1}{\nwixu{NW4GQOIU-SIrPU-1}\nwixd{NW4GQOIU-i3X8w-1}\nwixu{NW4GQOIU-fQI0z-1}\nwixu{NW4GQOIU-UTABh-1}\nwixu{NW4GQOIU-UTABh-4}}}%
\nwixlogsorted{c}{{distclean recipe}{NW4GQOIU-1Hidyl-1}{\nwixu{NW4GQOIU-LrkAl-1}\nwixd{NW4GQOIU-1Hidyl-1}}}%
\nwixlogsorted{c}{{document files}{NW4GQOIU-2cBbLf-1}{\nwixu{NW4GQOIU-4QwfwR-1}\nwixd{NW4GQOIU-2cBbLf-1}}}%
\nwixlogsorted{c}{{download rfc.bib}{NW4GQOIU-472ZSW-1}{\nwixu{NW4GQOIU-1nm7wo-1}\nwixd{NW4GQOIU-472ZSW-1}}}%
\nwixlogsorted{c}{{indices files}{NW4GQOIU-UTABh-1}{\nwixu{NW4GQOIU-4QwfwR-1}\nwixd{NW4GQOIU-UTABh-1}\nwixd{NW4GQOIU-UTABh-2}\nwixd{NW4GQOIU-UTABh-3}\nwixd{NW4GQOIU-UTABh-4}}}%
\nwixlogsorted{c}{{latexmkrc}{NW4GQOIU-3qE2M6-1}{\nwixd{NW4GQOIU-3qE2M6-1}\nwixd{NW4GQOIU-3qE2M6-2}}}%
\nwixlogsorted{c}{{PoPETS}{NW4GQOIU-347V4M-1}{\nwixu{NW4GQOIU-2Oad4t-1}\nwixd{NW4GQOIU-347V4M-1}}}%
\nwixlogsorted{c}{{PythonTeX files}{NW4GQOIU-287s1n-1}{\nwixu{NW4GQOIU-4QwfwR-1}\nwixd{NW4GQOIU-287s1n-1}}}%
\nwixlogsorted{c}{{Springer LNCS}{NW4GQOIU-42CEER-1}{\nwixu{NW4GQOIU-2Oad4t-1}\nwixd{NW4GQOIU-42CEER-1}\nwixd{NW4GQOIU-42CEER-2}}}%
\nwixlogsorted{c}{{target for cleaning}{NW4GQOIU-YyAiu-1}{\nwixu{NW4GQOIU-1Zq3gR-1}\nwixd{NW4GQOIU-YyAiu-1}}}%
\nwixlogsorted{c}{{target for latexmkrc}{NW4GQOIU-4Jzoxc-1}{\nwixu{NW4GQOIU-4QwfwR-1}\nwixd{NW4GQOIU-4Jzoxc-1}}}%
\nwixlogsorted{c}{{targets for archive}{NW4GQOIU-3iv42z-1}{\nwixu{NW4GQOIU-1Zq3gR-1}\nwixd{NW4GQOIU-3iv42z-1}\nwixd{NW4GQOIU-3iv42z-2}}}%
\nwixlogsorted{c}{{targets for class and package files}{NW4GQOIU-tejPg-1}{\nwixu{NW4GQOIU-2kaaDB-1}\nwixd{NW4GQOIU-tejPg-1}}}%
\nwixlogsorted{c}{{targets for cleaning}{NW4GQOIU-LrkAl-1}{\nwixu{NW4GQOIU-2kaaDB-1}\nwixd{NW4GQOIU-LrkAl-1}}}%
\nwixlogsorted{c}{{targets for desired files}{NW4GQOIU-22rOUj-1}{\nwixu{NW4GQOIU-1Zq3gR-1}\nwixd{NW4GQOIU-22rOUj-1}}}%
\nwixlogsorted{c}{{targets for documents}{NW4GQOIU-4QwfwR-1}{\nwixu{NW4GQOIU-2kaaDB-1}\nwixd{NW4GQOIU-4QwfwR-1}}}%
\nwixlogsorted{c}{{targets for external classes}{NW4GQOIU-2Oad4t-1}{\nwixu{NW4GQOIU-2kaaDB-1}\nwixd{NW4GQOIU-2Oad4t-1}}}%
\nwixlogsorted{c}{{targets for repo}{NW4GQOIU-4eSKbD-1}{\nwixu{NW4GQOIU-1Zq3gR-1}\nwixd{NW4GQOIU-4eSKbD-1}\nwixd{NW4GQOIU-4eSKbD-2}}}%
\nwixlogsorted{c}{{targets for symlinks}{NW4GQOIU-1DUTgs-1}{\nwixu{NW4GQOIU-1Zq3gR-1}\nwixd{NW4GQOIU-1DUTgs-1}\nwixd{NW4GQOIU-1DUTgs-2}}}%
\nwixlogsorted{c}{{tex.mk}{NW4GQOIU-2kaaDB-1}{\nwixd{NW4GQOIU-2kaaDB-1}}}%
\nwixlogsorted{c}{{the RFC bibliography}{NW4GQOIU-1nm7wo-1}{\nwixu{NW4GQOIU-2Oad4t-1}\nwixd{NW4GQOIU-1nm7wo-1}\nwixd{NW4GQOIU-1nm7wo-2}\nwixd{NW4GQOIU-1nm7wo-3}}}%
\nwixlogsorted{c}{{variables}{NW4GQOIU-29jjeg-1}{\nwixd{NW4GQOIU-29jjeg-1}\nwixd{NW4GQOIU-29jjeg-2}\nwixd{NW4GQOIU-29jjeg-3}\nwixd{NW4GQOIU-29jjeg-4}\nwixd{NW4GQOIU-29jjeg-5}\nwixd{NW4GQOIU-29jjeg-6}\nwixd{NW4GQOIU-29jjeg-7}\nwixd{NW4GQOIU-29jjeg-8}\nwixu{NW4GQOIU-2kaaDB-1}\nwixd{NW4GQOIU-29jjeg-9}}}%
\nwbegindocs{110}\nwdocspar
\nwenddocs{}
